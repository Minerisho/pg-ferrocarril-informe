% !TeX root = ../main.tex


\chapter{Metodología}

La metodología seleccionada para el desarrollo de este proyecto corresponde a una adaptación del Proceso Racional Unificado (RUP) complementado con tableros Kanban, permitiendo combinar un enfoque estructurado con la flexibilidad de los métodos ágiles.
El modelo RUP fue elegido debido a su enfoque iterativo e incremental, que permite refinar progresivamente el producto a través de fases bien definidas (Inicio, Elaboración, Construcción e Implementación), garantizando la trazabilidad y calidad del proceso.
Por su parte, Kanban proporciona una herramienta visual de gestión que facilita la organización de tareas, el control de avances y la asignación de responsabilidades, elementos fundamentales en un equipo de trabajo pequeño que requiere adaptabilidad y comunicación constante.

La integración de ambos enfoques permitió mantener la rigurosidad metodológica necesaria para un proyecto académico, sin perder la capacidad de respuesta ante imprevistos técnicos o de diseño durante la construcción del prototipo.


\section{Inicio}

La fase de inicio tuvo como propósito establecer las bases conceptuales, técnicas y pedagógicas del proyecto, en la cual se han abordado todos los conceptos utilizados en el artículo recopilatorio GBL \autocite{gblFrameworkExamining}. En esta etapa se definió la visión general del videojuego serio, sus objetivos formativos y las necesidades que este debía atender dentro del contexto del aprendizaje sobre sistemas ferroviarios en Colombia.

También se identificaron los principales riesgos técnicos y de diseño, así como las limitaciones de tiempo, recursos y experiencia del equipo, con el fin de planificar un alcance realista del prototipo

\subsection{Actividades a realizar}

\begin{enumerate}
  \item \textbf{Definir las HU:} Se define un primer documento borrador de las historias de usuario iniciales del videojuego para que puedan cumplir con los requisitos para ser un GBL.
  \item \textbf{Identificar riesgos:} Se definen en una lista las posibles dificultades técnicas encontradas en Unity para cumplir con las historias de usuario planteadas.
  \item \textbf{Planificar la fase de elaboración:} Se empieza a elaborar un primer borrador del documento de los requisitos clave funcionales.
\end{enumerate}

\section{Elaboración}

El objetivo principal de esta fase fue transformar la visión conceptual en un modelo técnico y funcional inicial, detallando los requerimientos, la arquitectura y el diseño general del videojuego.
Durante esta etapa se buscó reducir la incertidumbre técnica mediante la creación de maquetas y la priorización de historias de usuario.

\subsection{Actividades a realizar}

\begin{enumerate}
  \item \textbf{Organizar las HU:} Se pule el documento de las historias de usuario previamente encontradas y se organizan en épicas para después definir un orden de prioridades. Además, se explicará cómo se abordarán los criterios del GBL y los desafíos ferroviarios en las historias de usuario.
  \item \textbf{Elaborar maqueta:} Se construye una primera maqueta en Unity atacando directamente a la lista de identificación de riesgos, se obtiene un primer “esqueleto” del juego sobre el cual se empezará la fase de construcción.
  \item \textbf{Refinar la lista de riesgos:} Se modifican los riesgos según el resultado de la primera maqueta, para después investigar posibles soluciones.
  \item \textbf{Planificar las iteraciones:} Se definen las iteraciones que se darán en la fase de construcción, priorizando los requerimientos e historias de usuario más importantes.
\end{enumerate}

\section{Implementación}

Durante esta fase se llevó a cabo la programación y construcción del prototipo del videojuego de forma incremental, completando las funcionalidades y asegurando la calidad a través de pruebas continuas. La actividad principal de esta fase es implementar el videojuego y, a partir de las iteraciones planificadas, se empieza a construir el código en Unity apoyándose en el sistema de gestión de procesos Kanban. 

El apoyo del sistema de gestion Kanban permitió mantener una comunicación visual y fluida sobre el avance del proyecto, registrando las tareas en las columnas Por hacer, En proceso, En revisión y Hecho. Cada integrante era responsable de una o varias historias de usuario, y, una vez una tarea este en la columna "Revisandose" el compañero debía validar su funcionamiento antes de marcar la tarea como finalizada

\begin{table}[H]\centering
\caption{Tablero Kanban propuesto para la fase de Construcción.}
\label{tab:kanban-construccion}
\begin{tabular}{@{}lcccc@{}}
\toprule
\textbf{Equipo} & \textbf{Por hacer} & \textbf{Haciéndose} & \textbf{Revisándose} & \textbf{Hecho} \\
\midrule
Miguel & & & & \\
Mateo  & & & & \\
\bottomrule
\end{tabular}
\end{table}

\section{Evaluación}

La fase de evaluación tuvo como propósito validar el prototipo desde los aspectos técnicos, pedagógicos y de usabilidad. Para ello se diseñó una prueba piloto dirigida a un grupo de estudiantes universitarios de la Universidad Industrial de Santander (UIS), quienes interactuaron con el videojuego y respondieron a un cuestionario estructurado para medir la experiencia y el aprendizaje percibido

\subsection{Actividades a realizar}

\begin{enumerate}
  \item \textbf{Obtener y organizar grupo de prueba:} Se acuerda con un grupo de estudiantes UIS para acordar una fecha y lugar para realizar la prueba.
  \item \textbf{Obtener los resultados:} Una vez realizada la prueba se les realiza un pequeño cuestionario para evaluar el rendimiento, la usabilidad y las recomendaciones del jugador.
  \item \textbf{Preparar informe y presentación:} Recopilar la documentación, el proceso de trabajo y los resultados para agruparlos en un libro entregable junto con una presentación.
\end{enumerate}
