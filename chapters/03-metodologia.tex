% !TeX root = ../main.tex

\chapter{Metodología}

La metodología seleccionada para el desarrollo de este proyecto correspondió a una adaptación del Proceso Racional Unificado (RUP) complementado con tableros Kanban, permitiendo combinar un enfoque estructurado con la flexibilidad de los métodos ágiles.

El modelo RUP fue elegido debido a su enfoque iterativo e incremental, que permitió refinar progresivamente el producto a través de fases bien definidas (Inicio, Elaboración, Construcción e Implementación), garantizando la trazabilidad y calidad del proceso.

Por su parte, Kanban proporcionó una herramienta visual de gestión que facilitó la organización de tareas, el control de avances y la asignación de responsabilidades, elementos fundamentales en un equipo de trabajo pequeño que requería adaptabilidad y comunicación constante.

La integración de ambos enfoques permitió mantener la rigurosidad metodológica necesaria para un proyecto académico, sin perder la capacidad de respuesta ante imprevistos técnicos o de diseño durante la construcción del prototipo.


\section{Inicio}

La fase de inicio tuvo como propósito establecer las bases conceptuales, técnicas y pedagógicas del proyecto, en la cual se abordaron todos los conceptos utilizados en el artículo recopilatorio GBL \autocite{gblFrameworkExamining}{.} En esta etapa se definió la visión general del videojuego serio, sus objetivos formativos y las necesidades que este debía atender dentro del contexto del aprendizaje sobre sistemas ferroviarios en Colombia.

También se identificaron los principales riesgos técnicos y de diseño, así como las limitaciones de tiempo, recursos y experiencia del equipo, con el fin de planificar un alcance realista del prototipo

\subsection{Actividades realizadas}

\begin{enumerate}
  \item Definir las HU: Se definió un primer documento borrador de las historias de usuario iniciales del videojuego para que pudieran cumplir con los requisitos para ser un GBL.
  \item Identificar riesgos: Se definieron en una lista las posibles dificultades técnicas encontradas en Unity para cumplir con las historias de usuario planteadas.
  \item Planificar la fase de elaboración: Se empezó a elaborar un primer borrador del documento de los requisitos clave funcionales.
\end{enumerate}

\section{Elaboración}

El objetivo principal de esta fase fue transformar la visión conceptual en un modelo técnico y funcional inicial, detallando los requerimientos, la arquitectura y el diseño general del videojuego.
Durante esta etapa se buscó reducir la incertidumbre técnica mediante la creación de maquetas y la priorización de historias de usuario.

\subsection{Actividades realizadas}

\begin{enumerate}
  \item Organizar las HU: Se pulió el documento de las historias de usuario previamente encontradas y se organizaron en épicas para después definir un orden de prioridades. Además, se explicó cómo se abordarían los criterios del GBL y los desafíos ferroviarios en las historias de usuario.
  \item Elaborar maqueta: Se construyó una primera maqueta en Unity atacando directamente a la lista de identificación de riesgos, se obtuvo un primer “esqueleto” del juego sobre el cual se empezó la fase de construcción.
  \item Refinar la lista de riesgos: Se modificaron los riesgos según el resultado de la primera maqueta, para después investigar posibles soluciones.
  \item Planificar las iteraciones: Se definieron las iteraciones que se darían en la fase de construcción, priorizando los requerimientos e historias de usuario más importantes.
\end{enumerate}

\section{Implementación}

Durante esta fase se llevó a cabo la programación y construcción del prototipo del videojuego de forma incremental, completando las funcionalidades y asegurando la calidad a través de pruebas continuas. La actividad principal de esta fase fue implementar el videojuego y, a partir de las iteraciones planificadas, se empezó a construir el código en Unity apoyándose en el sistema de gestión de procesos Kanban. 

El apoyo del sistema de gestión Kanban permitió mantener una comunicación visual y fluida sobre el avance del proyecto, registrando las tareas en las columnas \textit{Por hacer} y \textit{Haciéndose}. Cada integrante era responsable de una o varias historias de usuario, y, una vez una tarea estaba en la columna \textit{Revisándose} el compañero debía validar su funcionamiento antes de marcar la tarea como \textit{Hecho}.

\begin{table}[H]\centering
\caption{Tablero Kanban propuesto para la fase de Construcción.}
\label{tab:kanban-construccion}
\begin{tabular}{@{}lcccc@{}}
\toprule
\textbf{Equipo} & \textbf{Por hacer} & \textbf{Haciéndose} & \textbf{Revisándose} & \textbf{Hecho} \\
\midrule
Miguel & & & & \\
Mateo  & & & & \\
\bottomrule
\end{tabular}
\end{table}

\section{Evaluación}

La fase de evaluación tuvo como propósito validar el prototipo desde los aspectos técnicos, pedagógicos y de usabilidad. Para ello se diseñó una prueba piloto dirigida a un grupo de estudiantes universitarios de la Universidad Industrial de Santander (UIS), quienes interactuaron con el videojuego y respondieron a un cuestionario estructurado para medir la experiencia y el aprendizaje percibido

\subsection{Actividades realizadas}

\begin{enumerate}
  \item Publicar videojuego serio: Se publicó y publicitó un link durante un periodo de 1 semana donde las personas pudieran instalar y probar fácilmente el videojuego.
  \item Obtener los resultados: Los resultados de juego y rendimiento de los participantes se guardaron en una base de datos. Además se les pidió realizar un pequeño cuestionario para calificar la usabilidad del software.
  \item Análisis de datos: Los datos recopilados pasaron a ser organizados y estudiados para extraer conclusiones.
  \item Preparar informe y presentación: Se recopiló la documentación, el proceso de trabajo y los resultados para agruparlos en un libro entregable junto con una presentación.
\end{enumerate}