% !TeX root = ../main.tex


\chapter{Metodología}

La metodología que se utilizará es RUP simplificada y adaptada con tableros Kanban, por ello, se adecuaron los objetivos del proyecto de la sección 2 del documento a las fases descritas en RUP.

\section{Inicio}

En esta fase, aunque ya avanzada por la investigación hecha para las entregas del tema y el plan de trabajo de grado, se aborda todos los conceptos utilizados en el artículo recopilatorio GBL \autocite{gblFrameworkExamining}. El objetivo de esta fase es definir un primer documento de historias de usuario y riesgos del prototipo, identificar los requisitos clave (funcionalidades GBL, desafíos ferroviarios a representar) a alto nivel. Las actividades son las siguientes:

\begin{enumerate}
  \item \textbf{Definir las HU:} Se define un primer documento borrador de las historias de usuario iniciales del videojuego para que puedan cumplir con los requisitos para ser un GBL.
  \item \textbf{Identificar riesgos:} Se definen en una lista las posibles dificultades técnicas encontradas en Unity para cumplir con las historias de usuario planteadas.
  \item \textbf{Planificar la fase de elaboración:} Se empieza a elaborar un primer borrador del documento de los requisitos clave funcionales.
\end{enumerate}

\section{Elaboración}

El objetivo de esta fase es mitigar los riesgos principales, agrupar y jerarquizar las HU, y definir y validar la arquitectura base del videojuego. Las actividades son las siguientes:

\begin{enumerate}
  \item \textbf{Organizar las HU:} Se pule el documento de las historias de usuario previamente encontradas y se organizan en épicas para después definir un orden de prioridades. Además, se explicará cómo se abordarán los criterios del GBL y los desafíos ferroviarios en las historias de usuario.
  \item \textbf{Elaborar maqueta:} Se construye una primera maqueta en Unity atacando directamente a la lista de identificación de riesgos, se obtiene un primer “esqueleto” del juego sobre el cual se empezará la fase de construcción.
  \item \textbf{Refinar la lista de riesgos:} Se modifican los riesgos según el resultado de la primera maqueta, para después investigar posibles soluciones.
  \item \textbf{Planificar las iteraciones:} Se definen las iteraciones que se darán en la fase de construcción, priorizando los requerimientos e historias de usuario más importantes.
\end{enumerate}

\section{Implementación}

El objetivo de esta fase es construir el prototipo del videojuego de forma incremental, completando las funcionalidades y asegurando la calidad a través de pruebas continuas. La actividad principal de esta fase es implementar el videojuego y, a partir de las iteraciones planificadas, se empieza a construir el código en Unity apoyándose en el sistema de gestión de procesos Kanban. Para esto se asignarán los requerimientos e historias de usuario a cada uno de los miembros del equipo y se empezará a avanzar en un tablero Kanban como el siguiente:

\begin{table}[H]\centering
\caption{Tablero Kanban propuesto para la fase de Construcción.}
\label{tab:kanban-construccion}
\begin{tabular}{@{}lcccc@{}}
\toprule
\textbf{Equipo} & \textbf{Por hacer} & \textbf{Haciéndose} & \textbf{Revisándose} & \textbf{Hecho} \\
\midrule
Miguel & & & & \\
Mateo  & & & & \\
\bottomrule
\end{tabular}
\end{table}

La columna “Revisándose” hará que el integrante contrario revise su historia de usuario y, una vez este lo pruebe y le dé su aval, pasa a considerarse como “Hecho”.

\section{Evaluación}

El objetivo de esta fase es entregar el prototipo a los usuarios de la prueba piloto (estudiantes UIS) y escribir el libro final entregable del trabajo de grado. Las actividades son las siguientes:

\begin{enumerate}
  \item \textbf{Obtener y organizar grupo de prueba:} Se acuerda con un grupo de estudiantes UIS para acordar una fecha y lugar para realizar la prueba.
  \item \textbf{Obtener los resultados:} Una vez realizada la prueba se les realiza un pequeño cuestionario para evaluar el rendimiento, la usabilidad y las recomendaciones del jugador.
  \item \textbf{Preparar informe y presentación:} Recopilar la documentación, el proceso de trabajo y los resultados para agruparlos en un libro entregable junto con una presentación.
\end{enumerate}
