\chapter{Conclusiones y trabajo futuro}\label{chap:conclusiones}
\section{Conclusiones}
recuerden que la experiencia es importante
% Conclusiones principales alineadas con los objetivos.
% Perspectivas: ampliación de escenarios, más eventos del dominio, evaluación
% con muestras más grandes, integración de analíticas de aprendizaje, etc.

\section{Mejoras y Trabajo Futuro}

Para expandir el alcance del videojuego serio, se proponen las siguientes líneas de trabajo:
\begin{enumerate}
    \item \textbf{Expansión del contenido histórico:} Incluir más narrativa sobre la historia real de los Ferrocarriles Nacionales de Colombia, desbloqueable mediante coleccionables en el mapa y mediante dialogos proporcionados por Don Raíl.
    \item \textbf{Expansión del contenido jugable:} Incorporar niveles adicionales con una progresión de dificultad más gradual, de modo que la transición entre etapas sea más equilibrada y se eviten saltos bruscos en el reto para el jugador.
    \item \textbf{Modo "Sandbox":} Habilitar un modo sin restricciones económicas para fomentar la creatividad libre del usuario.
    \item \textbf{Portabilidad WebGL:} Optimizar los recursos gráficos para permitir la ejecución del juego directamente en navegadores web, facilitando su acceso en entornos educativos sin necesidad de instalación.
    \item \textbf{Mejoras de calidad de vida:} Implementar diversas optimizaciones orientadas a la experiencia del usuario, tales como la inclusión de un cronómetro al inicio del recorrido del tren, una mayor visibilidad en las zonas de daño, opciones de configuración de pantalla que permitan elegir entre modo ventana o pantalla completa, así como la incorporación de atajos de teclado para agilizar acciones frecuentes (por ejemplo, utilizar \texttt{Ctrl+Z} para borrar el ultimo tramo creado).
    \item \textbf{Mejora de tutorial: } Aumentar y mejorar la calidad de los tutoriales impartidos por Don Raíl, con el objetivo de que más gente comprenda correctamente el funcionamiento y manejo del sistema.
    \item \textbf{Descarga desde plataformas confiables: } Publicar el juego en sitios reconocidos de distribución digital, como \textit{Steam} o \textit{itch.io}, con el fin de facilitar el acceso a las descargas y permitir que un mayor número de usuarios lo pruebe.
    \item \textbf{IP e identificador en la base de datos: } Incorporar en la base de datos los campos correspondientes a la dirección IP y al nombre de usuario, con el propósito de optimizar la retroalimentación hacia el jugador y facilitar la consulta de su información.
\end{enumerate}