\chapter{Conclusiones}\label{chap:conclusiones}

Tras la realizacion del prototipo del videojuego serio, y la realizacion de las pruebas, se pueden determinar las siguientes conclusiones

\begin{enumerate}
    \item Los criterios GBL fueron una buena estandarización de lo que es realmente un juego educativo, y ayudó mucho a fijar objetivos y artefactos claros como fue el caso de las historias de usuario.
    \item La estructura y la metodología usadas durante todo el proyecto ayudaron favorablemente al ritmo del ciclo de vida del videojuego serio. Ya que, a pesar de haber dificultades técnicas y retrasos durante el ciclo, los artefactos generados ayudaron a recuperar el trayecto facilmente.
    \item La arquitectura de escenas aditivas para el desarrollo del videojuego serio en Unity fue una decisión diferenciadora ya que permitió una mejora del rendimiento de carga de escenas y proporcionó una mejor organización de las mismas.
    \item Durante las pruebas de usuarios se observó que los jugadores tienden a jugar con la cámara alejada del terreno, esto puede indicar muchas cosas como, la falta de poder ubicar más facilmente los municipios, la molestia por tener que moverse tanto para hacer la via, o se les hace más cómodo tener una vista amplia del terreno.
    \item Según la  \cref{fig:retencion}, 28 usuarios que probaron el videojuego serio, pasaron por el nivel 1, 9 personas pasaron por el nivel 2, 8 personas pasaron por el nivel 3, 5 personas pasaron por el nivel 4 y finalmente 1 persona pasó por el nivel 5, esto demuestra una disminución en la retención de los jugadores.
    \item Como se puede observar en la  \cref{fig:tasa_exito} la proporción de victorias frente a derrotas por nivel demuestra la curva de dificultad, siendo el nivel 1 un nivel con un alto índice de derrota al ser un nivel introductorio, mientras que el nivel 2 demuestra una reducción considerable de dicho índice y a medida que pasan los niveles la dificultad aumenta, hasta normalizarse en el nivel 4.
    \item En la pregunta de la \cref{fig:encuestaaprendizaje}, las respuestas demostraron que resultados de la prueba piloto evidenciaron una concientización de la problemática ferroviaria Colombiana, además, tal como se vio en la \cref{fig:contruisviaslejanas}, el 81,8 \% de los participantes identificó la dificultad en la infrastructura férrea como una tarea compleja y costosa debido al impacto de la topografía y el uso necesario de túneles y puentes.
    \item La identificación de los túneles y puentes como los mayores focos de gasto ( \cref{fig:dineroyesfuerzo}) confirma que el usuario comprendió el impacto de la topografía. Finalmente, las respuestas abiertas ( \cref{fig:encuestaaprendizaje}) reflejan que los jugadores ahora valoran el desarrollo férreo no solo como un deseo político, sino como un reto de ingeniería de gran magnitud, cumpliendo así el objetivo principal de este proyecto de grado.
    \item En las respuestas de la pregunta 9 de la  \cref{fig:facilidadcontroles} (¿Que tan fáciles fueron de entender los controles?) se observó que el porcentaje de jugadores que manifestaron haber tenido problemas con los controles, son el mismo porcentaje que manifestaron haber tenido una confusa primera experiencia de juego en la pregunta 4 de la  \cref{fig:primeraexperiencia}.
\end{enumerate}