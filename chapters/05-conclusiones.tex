\chapter{Conclusiones y trabajo futuro}\label{chap:conclusiones}
\section{Conclusiones}
\begin{enumerate}
    \item Durante las pruebas de usuarios se observo que los jugadores tienden a jugar con la cámara alejada del terreno, esto puede indicar muchas cosas como, la falta de poder ubicar más facilmente los municipios, la molestia por tener que moverse tanto para hacer la via, o se les hace más comodo tener una vista amplia del terreno.
    \item La estructura y la metodología usadas durante todo el proyecto ayudaron favorablemente al ritmo del ciclo de vida del videojuego serio. Ya que, a pesar de haber dificultades técnicas y retrasos durante el ciclo, los artefactos generados ayudaron a recuperar el trayecto facilmente.
    \item Los criterios GBL fueron una buena estandarización de lo que es realmente un juego educativo, y ayudó mucho a fijar objetivos y artefactos claros como fue el caso de las historias de usuario.
    \item La arquitectura de escenas aditivas para el desarrollo del videojuego serio en Unity fue una decisión diferenciadora ya que permitió una mejora del rendimiento de carga de escenas y proporcionó una mejor organización de las mismas.
    \item Segun la Figura \ref{fig:retencion}, 28 usuarios que probaron el videojuego serio, pasaron por el nivel 1, 9 personas pasaron por el nivel 2, 8 personas pasaron por el nivel 3, 5 personas pasaron por el nivel 4 y finalmente 1 persona pasó por el nivel 5, esto demuestra una disminucion en la retención de los jugadores.
    \item Como se puede observar en la Figura \ref{fig:tasa_exito} la proporcion de victorias frente a derrotas por nivel demuestra la curva de dificultad, siendo el nivel 1 un nivel con un alto indice de derrota al ser un nivel introductorio, mientras que el nivel 2 demuestra una reducción considerable de dicho indice y a medida que pasan los niveles la dificultad aumenta, hasta normalizarse en el nivel 4.
    \item En la pregunta de la Figura \ref{fig:encuestaaprendizaje}, las respuestas demostraron.
    \item La identificación de los túneles y puentes como los mayores focos de gasto (Figura \ref{fig:dineroyesfuerzo}) confirma que el usuario comprendió el impacto de la topografía. Finalmente, las respuestas abiertas (Figura \ref{fig:encuestaaprendizaje}) reflejan que los jugadores ahora valoran el desarrollo férreo no solo como un deseo político, sino como un reto de ingeniería de gran magnitud, cumpliendo así el objetivo principal de este proyecto de grado.
    \item En las respuestas de la pregunta 9 de la Figura \ref{fig:facilidadcontroles} (¿Que tan faciles fueron de entender los controles?) se observo que el porcentaje de jugadores que manifestaron haber tenido problemas con los controles, son el mismo porcentaje que manifestaron haber tenido una confusa primera experiencia de juego en la pregunta 4 de la Figura \ref{fig:primeraexperiencia}.
\end{enumerate}

\section{Mejoras y Trabajo Futuro}

Para expandir el alcance del videojuego serio, se proponen las siguientes líneas de trabajo:
\begin{enumerate}
    \item \textbf{Expansión del contenido histórico:} Incluir más narrativa sobre la historia real de los Ferrocarriles Nacionales de Colombia, desbloqueable mediante coleccionables en el mapa y mediante dialogos proporcionados por Don Raíl.
    \item \textbf{Expansión del contenido jugable:} Incorporar niveles adicionales con una progresión de dificultad más gradual, de modo que la transición entre etapas sea más equilibrada y se eviten saltos bruscos en el reto para el jugador.
    \item \textbf{Modo "Sandbox":} Habilitar un modo sin restricciones económicas para fomentar la creatividad libre del usuario.
    \item \textbf{Portabilidad WebGL:} Optimizar los recursos gráficos para permitir la ejecución del juego directamente en navegadores web, facilitando su acceso en entornos educativos sin necesidad de instalación.
    \item \textbf{Mejoras de calidad de vida:} Implementar diversas optimizaciones orientadas a la experiencia del usuario, tales como la inclusión de un cronómetro al inicio del recorrido del tren, una mayor visibilidad en las zonas de daño, opciones de configuración de pantalla que permitan elegir entre modo ventana o pantalla completa, así como la incorporación de atajos de teclado para agilizar acciones frecuentes (por ejemplo, utilizar \texttt{Ctrl+Z} para borrar el ultimo tramo creado).
    \item \textbf{Mejora de tutorial: } Aumentar y mejorar la calidad de los tutoriales impartidos por Don Raíl, con el objetivo de que más gente comprenda correctamente el funcionamiento y manejo del sistema.
    \item \textbf{Descarga desde plataformas confiables: } Publicar el juego en sitios reconocidos de distribución digital, como \textit{Steam} o \textit{itch.io}, con el fin de facilitar el acceso a las descargas y permitir que un mayor número de usuarios lo pruebe.
    \item \textbf{IP e identificador en la base de datos: } Incorporar en la base de datos los campos correspondientes a la dirección IP y al nombre de usuario, con el propósito de optimizar la retroalimentación hacia el jugador y facilitar la consulta de su información.
    \item \textbf{Informes desde la web: } Implementar un sistema de elaboración de informes a través de una página web, con el fin de facilitar su generación y acceso.
\end{enumerate}