% !TeX root = ../main.tex

\chapter{Resultados}

En este capítulo se presentan los productos resultantes del proceso de desarrollo del prototipo del videojuego serio, abarcando desde la consolidación de los requerimientos y el diseño de la arquitectura, hasta la implementación del prototipo y el análisis de los datos obtenidos durante la prueba piloto.

\section{Requerimientos}

Los artefactos presentados a continuación son el producto de un proceso de refinamiento continuo llevado a cabo durante las fases de inicio y elaboración. Tras múltiples ciclos de iteración y ajuste, se obtuvieron estos documentos que constituyen la versión final y consolidada de los requisitos funcionales y el análisis de riesgos que guiaron la construcción del software.

\subsection{Matriz de Riesgos}

La gestión de riesgos fue un proceso continuo durante el desarrollo. Para la evaluación final, se utilizó la escala de medición definida en la etapa de planificación, donde el nivel de riesgo se determina por el producto del Impacto por la Probabilidad de ocurrencia (ver Figura \ref{fig:escala_riesgos}).
\begin{figure}[H]
\centering
\includegraphics[width=1\textwidth]{figures/linea de impacto-probabilidad.png}
\caption{Escala de evaluación de riesgos utilizada durante el proyecto.}
\label{fig:escala_riesgos}
\end{figure}

La Tabla \ref{tab:matriz-riesgos-final} presenta la matriz de riesgos. Es importante destacar que, tras las iteraciones de desarrollo y las correcciones implementadas en la fase de implementación, todos los riesgos técnicos críticos fueron mitigados y cerrados exitosamente antes de la prueba piloto. La columna \textit{Estado} detalla la solución final aplicada para cada caso.

% Ajuste de ancho de columnas para acomodar el texto descriptivo
\begin{landscape}
\begin{longtable}{|c|L{2.5cm}|L{2.5cm}|c|c|c|L{3cm}|L{4.5cm}|}
\caption{Matriz de riesgos del proyecto.}\label{tab:matriz-riesgos-final} \\
\hline
\multicolumn{1}{|c|}{\textbf{Ref}} &
\multicolumn{1}{c|}{\textbf{Descripción}} &
\multicolumn{1}{c|}{\textbf{Causa}} &
\multicolumn{1}{c|}{\textbf{Imp.}} &
\multicolumn{1}{c|}{\textbf{Prob.}} &
\multicolumn{1}{c|}{\textbf{Riesgo}} &
\multicolumn{1}{c|}{\textbf{Plan de acción}} &
\multicolumn{1}{c|}{\textbf{Estado}} \\
\hline
\endfirsthead

\hline
\multicolumn{1}{|c|}{\textbf{Ref}} &
\multicolumn{1}{c|}{\textbf{Descripción}} &
\multicolumn{1}{c|}{\textbf{Causa}} &
\multicolumn{1}{c|}{\textbf{Imp.}} &
\multicolumn{1}{c|}{\textbf{Prob.}} &
\multicolumn{1}{c|}{\textbf{Riesgo}} &
\multicolumn{1}{c|}{\textbf{Plan de acción}} &
\multicolumn{1}{c|}{\textbf{Estado}} \\
\hline
\endhead

\hline
\multicolumn{8}{r}{\textit{Continúa en la siguiente página}} \\
\endfoot

\hline
\endlastfoot

R3 & Vías poco naturales e inconexas & Uso de rectas para generar vías & 5 & 4 & 20 & Implementación de curvas Bézier y Splines. & Cerrado. Se implementó la herramienta Dreamteck Splines para generar curvas suaves y continuas. \\ \hline

R1 & El modelo del terreno está mal optimizado & Sobrecarga en memoria gráfica por modelo 3D denso & 4 & 4 & 16 & Uso de Terrenos nativos de Unity y carga por bloques (LOD). & Cerrado. Migración completa al sistema de Terrenos de Unity con optimización LOD. \\ \hline

R7 & Dificultad para representar hidrografía completa & Densidad excesiva de ríos en Colombia & 4 & 4 & 16 & Selección de ríos principales y lagos mayores únicamente. & Cerrado. Se representaron solo los cuerpos de agua principales mediante texturas y shaders, sin colisión física compleja. \\ \hline

R13 & Bajos FPS en equipos de gama baja & Desarrollo en equipos de gama alta & 5 & 3 & 15 & Optimización de draw calls y reducción de polígonos. & Cerrado. Se añadieron opciones de calidad gráfica que permiten rendimiento estable en equipos de especificaciones medias. \\ \hline

R8 & Las vías atraviesan montañas & Falta de interpolación de alturas & 3 & 4 & 12 & Generación automática de túneles al detectar colisión con terreno. & Cerrado. El sistema instancia automáticamente prefabs de túneles cuando la vía intersecta el terreno. \\ \hline

R2 & La cámara atraviesa el terreno & Falta de colisión en la cámara & 2 & 5 & 10 & Implementación de Clamping y colisionadores para la cámara. & Cerrado. Se restringieron las coordenadas de la cámara para impedir que cruce la malla del suelo. \\ \hline

R12 & Incompatibilidad de Shaders y URP & Configuración por defecto de Unity & 3 & 3 & 9 & Ajuste manual de materiales para el Universal Render Pipeline. & Cerrado. Todos los materiales fueron actualizados para ser compatibles con URP. \\ \hline

R9 & Estadísticas de locomotoras no leídas & Datos dispersos en hijos del GameObject & 4 & 2 & 8 & Unificar la lógica en el script controlador padre. & Cerrado. Se centralizó la gestión de estadísticas en un único controlador en la raíz del objeto. \\ \hline

R4 & Terreno con parches o vacíos & DEM con datos erróneos & 2 & 3 & 6 & Corrección mediante algoritmos de suavizado. & Cerrado. Corrección manual de imperfecciones utilizando las herramientas de escultura del editor. \\ \hline

R6 & Circuito cerrado no detectado & Spline no cerrado & 2 & 3 & 6 & Detección de proximidad de nodos inicial/final. & Cerrado. Validación por código que cierra el spline automáticamente al acercar los extremos. \\ \hline

R10 & HUD no muestra vida tras cambio & Referencias perdidas al cambiar locomotora & 2 & 2 & 4 & Función para reiniciar estadísticas al instanciar. & Cerrado. Se fuerza la actualización de la UI cada vez que se genera un nuevo tren. \\ \hline

R11 & Modelos 3D con escalas erróneas & Exportación incorrecta desde Blender & 1 & 3 & 3 & Aplicar escalas y rotaciones antes de exportar. & Cerrado. Normalización de escala (1,1,1) en todos los modelos importados. \\ \hline

R5 & Textura de vías solapada & Vía al mismo nivel del suelo & 1 & 2 & 2 & Mover las vías generadas una pequeña distancia hacia arriba. & Cerrado. Se aplicó un offset vertical de 0.05 unidades a la malla de la vía. \\ \hline

\end{longtable}
\end{landscape}

\subsection{Historias de Usuario (HU)}

Las funcionalidades del videojuego se consolidaron en un conjunto de Historias de Usuario, priorizadas mediante la técnica MoSCoW (Must, Should, Could, Won't) para asegurar el cumplimiento de los objetivos pedagógicos mínimos viables. La Tabla \ref{tab:HU-final} muestra el listado final implementado.

\begin{landscape}
\begin{longtable}{|M{1.2cm}|M{1.5cm}|L{9cm}|M{2cm}|M{1.5cm}|}
\caption{Historias de Usuario Finales e Implementadas.} \label{tab:HU-final} \\
\hline
\multicolumn{1}{|c|}{\textbf{ÉPICA}} &
\multicolumn{1}{c|}{\textbf{ID}} &
\multicolumn{1}{c|}{\textbf{HISTORIA DE USUARIO}} &
\multicolumn{1}{c|}{\textbf{PRIORIDAD}} &
\multicolumn{1}{c|}{\textbf{COMPL.}} \\ \hline
\endfirsthead

\hline
\multicolumn{1}{|c|}{\textbf{ÉPICA}} &
\multicolumn{1}{c|}{\textbf{ID}} &
\multicolumn{1}{c|}{\textbf{HISTORIA DE USUARIO}} &
\multicolumn{1}{c|}{\textbf{PRIORIDAD}} &
\multicolumn{1}{c|}{\textbf{COMPL.}} \\ \hline
\endhead

\hline
\multicolumn{5}{r}{\textit{Continúa en la siguiente página}} \\
\endfoot

\hline
\endlastfoot

1 & HU1  & Como jugador, quiero ver una pantalla de selección de niveles para elegir cuál jugar. & M & 2 \\ \hline
1 & HU2  & Como jugador, quiero ver un contrato con presupuesto y objetivos al iniciar. & M & 1 \\ \hline
2 & HU3  & Como jugador, quiero mover la cámara para explorar el mapa. & M & 1 \\ \hline
2 & HU4  & Como jugador, quiero hacer zoom para ver detalles. & M & 1 \\ \hline
2 & HU6  & Como jugador, quiero visualizar el terreno 3D de Colombia. & M & 4 \\ \hline
3 & HU10 & Como jugador, quiero construir tramos de vía entre dos puntos. & M & 2 \\ \hline
3 & HU12 & Como jugador, quiero que se generen túneles automáticamente en montañas. & M & 4 \\ \hline
3 & HU13 & Como jugador, quiero construir estaciones para definir inicio y fin. & M & 3 \\ \hline
3 & HU15 & Como jugador, quiero ver mi presupuesto actualizado en tiempo real. & M & 2 \\ \hline
4 & HU17 & Como jugador, quiero seleccionar y comprar locomotoras. & M & 3 \\ \hline
4 & HU18 & Como jugador, quiero añadir vagones de carga o pasajeros. & M & 2 \\ \hline
5 & HU20 & Como jugador, quiero ver alertas visuales si el tren sufre daños. & M & 2 \\ \hline
6 & HU23 & Como jugador, quiero recibir una calificación basada en mi desempeño. & M & 3 \\ \hline
6 & HU24 & Como jugador, quiero que la puntuación considere la rentabilidad y durabilidad. & M & 3 \\ \hline
8 & HU29 & Como nuevo jugador, quiero un tutorial interactivo con \textit{Don Raíl}. & S & 4 \\ \hline
8 & HU30 & Como jugador, quiero recibir advertencias sobre pendientes pronunciadas. & S & 4 \\ \hline
6 & HU36 & Como administrador, quiero guardar metadatos de las partidas en una base de datos. & M & 3 \\ \hline

\end{longtable}
\end{landscape}

\section{Diseño del Sistema}

Para garantizar la escalabilidad y el cumplimiento de los requisitos educativos, se diseñó una arquitectura de software que integra el motor de videojuego con servicios externos de gestión de datos.

\subsection{Diagrama de Componentes}

El diseño interno del videojuego se estructuró bajo un enfoque modular dentro de Unity, separando la lógica de simulación, la interfaz de usuario y la gestión de datos. A continuación, se describen los componentes principales ilustrados en la Figura \ref{fig:diagrama_componentes}:

\begin{figure}[H]
\centering
% \includegraphics[width=0.9\textwidth]{figures/diagrama_componentes.png} % DESCOMENTAR Y AGREGAR IMAGEN
\framebox[0.9\textwidth]{\rule{0pt}{8cm} [INSERTAR DIAGRAMA DE COMPONENTES AQUÍ]}
\caption{Diagrama de componentes del videojuego serio.}
\label{fig:diagrama_componentes}
\end{figure}

\begin{itemize}
    \item \textbf{GameManager:} Es el núcleo orquestador del ciclo de vida del juego. Administra los estados (Menú, Construcción, Simulación, Pausa, Resultados) y coordina la comunicación entre los demás subsistemas.
    \item \textbf{Sistema de Construcción (Spline System):} Encargado de la lógica matemática para la generación de curvas Bézier, validación de pendientes y la instanciación de mallas procedimentales para vías, túneles y puentes.
    \item \textbf{Simulador de Física (Train Controller):} Gestiona el movimiento del tren a lo largo del spline, calculando velocidad, aceleración y detectando colisiones con zonas de evento o descarrilamientos.
    \item \textbf{Interfaz de Usuario (UI Manager):} Controla todos los elementos visuales en pantalla (HUD), incluyendo el feedback del tutorial de Don Raíl, los paneles de contrato y las pantallas de victoria/derrota.
    \item \textbf{Gestor de Datos (Data Handler):} Módulo responsable de serializar la información de la partida y comunicarse con la API externa para el registro de métricas.
\end{itemize}

\subsection{Arquitectura del Sistema y API de Informes}

Dado que uno de los objetivos del proyecto es evaluar el aprendizaje y comportamiento de los jugadores, se implementó una arquitectura Cliente-Servidor. El cliente (el videojuego desarrollado en Unity) se comunica con un servidor backend encargado de procesar y almacenar la telemetría de las sesiones.

La Figura \ref{fig:arquitectura_api} ilustra el flujo de información:

\begin{figure}[H]
\centering
% \includegraphics[width=0.9\textwidth]{figures/arquitectura_sistema.png} % DESCOMENTAR Y AGREGAR IMAGEN
\framebox[0.9\textwidth]{\rule{0pt}{8cm} [INSERTAR DIAGRAMA DE ARQUITECTURA + API AQUÍ]}
\caption{Arquitectura del sistema e integración con la API de informes.}
\label{fig:arquitectura_api}
\end{figure}

\begin{enumerate}
    \item \textbf{Cliente Unity (C\#):} Al finalizar un nivel, el cliente recopila métricas clave (tiempo, presupuesto usado, daños, puntuación) y empaqueta estos datos en un objeto JSON.
    \item \textbf{API REST:} Se desarrolló una API intermedia que recibe las peticiones POST provenientes del videojuego. Esta API valida la integridad de los datos y gestiona la seguridad de las transacciones.
    \item \textbf{Base de Datos:} Los registros se almacenan en una base de datos relacional/no-relacional [ESPECIFICAR TIPO, EJ: SQL/MongoDB], permitiendo la persistencia histórica de los intentos de los jugadores.
    \item \textbf{Módulo de Generación de Informes:} Un servicio independiente consulta la base de datos para generar visualizaciones estadísticas y reportes de rendimiento, los cuales son utilizados para la evaluación pedagógica presentada más adelante en este capítulo.
\end{enumerate}


\section{Implementación y Desarrollo}

La fase de implementación se llevó a cabo utilizando el motor Unity (versión 6.0), haciendo uso de Git para el control de versiones y LFS para la gestión de activos pesados. A continuación, se detalla el proceso técnico de construcción del entorno y las mecánicas ferroviarias.

\subsection{Configuración del Espacio de Trabajo en Unity}

Para el desarrollo del proyecto se configuró un entorno de trabajo optimizado en Unity 6.0, diseñado para gestionar la complejidad del mapa de Colombia y la lógica de los sistemas ferroviarios. En la Figura \ref{fig:interfaz_unity} se observa la disposición de las herramientas y la interacción entre las ventanas del editor:

\begin{enumerate}
    \item \textbf{Jerarquía (Hierarchy):} En esta sección se evidencia el uso de un flujo de trabajo basado en múltiples escenas cargadas simultáneamente (\textit{Multi-Scene Editing}), como \textit{MapaCompleto}, \textit{Nivel 1} y \textit{Nivel 2}. Cada escena agrupa sus propios componentes y lógica de forma independiente. En este caso, se encuentra desplegado el \textbf{Nivel 2}, donde se ha seleccionado el objeto \textbf{Locomotora}, permitiendo una edición focalizada sin alterar el resto del entorno global.
    
    \item \textbf{Vista de Escena y Juego (Scene \& Game View):} El área central permite alternar entre la edición y la visualización final. La pestaña activa (\textit{Scene}) muestra el espacio tridimensional donde se realiza el ensamblaje del relieve y la ubicación de objetos. Adyacente a esta, se encuentra la pestaña \textit{Game}, la cual permite previsualizar la cámara del jugador y la interfaz de usuario final para realizar pruebas de ejecución inmediatas.
    
    \item \textbf{Inspector:} Esta ventana actúa de forma reactiva a la selección realizada en la Jerarquía. Al estar señalada la \textbf{Locomotora}, el Inspector despliega todos los componentes asignados a dicho objeto. Desde aquí se gestionan propiedades como el \textit{Transform}, el sistema de audio y las variables físicas del motor del tren, permitiendo ajustar el comportamiento del vehículo en tiempo real.
    
    \item \textbf{Explorador de Proyecto (Project Window):} Contiene la biblioteca de archivos del juego organizada mediante una estructura de carpetas estandarizada (\textit{Meshes, Prefabs, Scripts, Settings}). Esta organización es fundamental para el control de versiones, asegurando que todos los activos del mapa de Colombia y los modelos ferroviarios estén correctamente indexados.
\end{enumerate}

\begin{figure}[H]
\centering
\includegraphics[width=1\textwidth]{figures/interfaz_unity.jpeg}
\caption{Interfaz de desarrollo en Unity: 1) Jerarquía con edición multiescena, 2) Pestañas de Escena y Juego, 3) Inspector de componentes de la locomotora, 4) Gestión de activos.}
\label{fig:interfaz_unity}
\end{figure}

\subsection{Construcción del Entorno Geográfico}

El mapa de Colombia representa el tablero de juego principal. Su implementación pasó por múltiples iteraciones para equilibrar el realismo geográfico con el rendimiento técnico.

Inicialmente, se importó un modelo 3D de alta resolución generado en Blender a partir de mapas de elevación (DEM). Como se observa en la Figura \ref{fig:mapa_inicial}, aunque visualmente detallado, este enfoque resultó inviable debido a la carga de 24.5 millones de vértices, lo que redujo la tasa de refresco a menos de 26 FPS.

Tras experimentar con versiones *low-poly* y fragmentadas que no solucionaron los problemas de *draw-calls*, se optó por la solución definitiva: el **Sistema de Terrenos de Unity**. Esta herramienta permitió:
\begin{itemize}
    \item Esculpir el relieve de la cordillera de los Andes con precisión topográfica pero optimizada mediante LOD (*Level of Detail*).
    \item Pintar texturas según la altura y la biome (selva, montaña, llanura).
    \item Integrar cuerpos de agua y vegetación sin sacrificar el rendimiento, alcanzando una tasa estable de 60 FPS en equipos de gama media.
\end{itemize}

La Figura \ref{fig:mapa_unity} muestra el resultado final del entorno implementado.

\begin{figure}[H]
\centering
\includegraphics[width=0.7\textwidth]{figures/mapa_unity.jpeg}
\caption{Implementación final del entorno geográfico utilizando el sistema de Terrenos de Unity.}
\label{fig:mapa_unity}
\end{figure}

\subsection{Sistema Ferroviario y Físicas}

La mecánica central del juego es la construcción y simulación de trenes. Para evitar la rigidez de un sistema basado en tramos rectos, se implementó una solución basada en **Splines** (curvas matemáticas suaves).

Se utilizó el paquete \textit{Dreamteck Splines} como base matemática, sobre la cual se programó la lógica de negocio del juego:
\begin{enumerate}
    \item \textbf{Generación Procedimental:} Al trazar una ruta, el sistema instancia automáticamente durmientes, rieles y balasto adaptándose a la curvatura del spline.
    \item \textbf{Lógica de Túneles y Puentes:} Mediante *Raycasting*, el sistema detecta si la vía atraviesa una montaña o un abismo. Si detecta terreno por encima de la vía, reemplaza el modelo de vía estándar por un túnel; si no detecta terreno por debajo, instancia pilares para crear un puente.
    \item \textbf{Movimiento del Tren:} El tren no utiliza físicas de ruedas complejas (que son costosas computacionalmente), sino que se proyecta a lo largo del spline. Sin embargo, se simulan variables físicas como la inercia y la fricción. La velocidad se calcula en función de la pendiente:
    \begin{equation}
        Vel_{actual} = Vel_{motor} - (Inclinacion \times Factor_{carga} \times Num_{vagones})
    \end{equation}
    Esto obliga al jugador a pensar como ingeniero: si la pendiente es muy alta y el tren lleva mucha carga, la velocidad puede llegar a cero, impidiendo completar el nivel.
\end{enumerate}

Para la representación visual, se modelaron trenes propios (Figura \ref{fig:tren_modelo}) y se implementó un sistema de daño visual (explosiones y humo) que se activa cuando el tren pasa por zonas de riesgo a alta velocidad (Figura \ref{fig:trenExplota}).

\begin{figure}[H]
\centering
\begin{minipage}{0.45\textwidth}
    \centering
    \includegraphics[width=\linewidth]{figures/tren_modelo.jpeg}
    \caption{Modelado de locomotoras.}
    \label{fig:tren_modelo}
\end{minipage}\hfill
\begin{minipage}{0.45\textwidth}
    \centering
    \includegraphics[width=\linewidth]{figures/trenExplota.png}
    \caption{Sistema de partículas de daño.}
    \label{fig:trenExplota}
\end{minipage}
\end{figure}

\section{Pruebas y Análisis de Resultados}

Para validar la efectividad del prototipo como herramienta de aprendizaje y su calidad como software, se realizó una prueba piloto controlada.
La fase de pruebas tuvo una duración de \textbf{4 días}, contando con la participación voluntaria de \textbf{22} personas de la comunidad universitaria.

El proceso de recolección de datos fue híbrido:
\begin{itemize}
    \item \textbf{Cuantitativo:} Mediante la API de informes, que registró automáticamente las estadísticas de juego de cada sesión.
    \item \textbf{Cualitativo:} Mediante una encuesta de percepción aplicada al finalizar la experiencia.
\end{itemize}

\subsection{Análisis de Telemetría (Datos de la API)}

Gracias a la arquitectura implementada, se pudo analizar el desempeño real de los jugadores sin depender únicamente de su autopercepción. A continuación, se presentan los hallazgos basados en los datos almacenados en la base de datos.

\subsubsection{Tasas de Éxito y Fracaso por Nivel}
La Figura \ref{fig:grafica_exito} muestra el porcentaje de intentos exitosos frente a los fallidos en cada uno de los niveles.

\begin{figure}[H]
\centering
% \includegraphics[width=0.8\textwidth]{figures/grafica_tasa_exito.png} % AGREGAR TU GRAFICA AQUI
\framebox[0.8\textwidth]{\rule{0pt}{6cm} [INSERTAR GRÁFICA DE BARRAS: ÉXITO VS FRACASO]}
\caption{Tasa de completitud por nivel registrada por la API.}
\label{fig:grafica_exito}
\end{figure}

Se observa que los primeros niveles tienen una tasa de éxito superior al 90\%, lo que valida la curva de aprendizaje inicial. Sin embargo, en el Nivel [X] se evidencia un pico de fracasos, correlacionado con la introducción de la mecánica de gestión de presupuesto limitado.

\subsubsection{Eficiencia de Costos y Tiempo}
La Figura \ref{fig:grafica_dispersion} relaciona el tiempo empleado vs. el presupuesto gastado por los jugadores que completaron el juego.

\begin{figure}[H]
\centering
% \includegraphics[width=0.8\textwidth]{figures/grafica_dispersion.png} % AGREGAR TU GRAFICA AQUI
\framebox[0.8\textwidth]{\rule{0pt}{6cm} [INSERTAR GRÁFICA DE DISPERSIÓN: TIEMPO VS COSTO]}
\caption{Relación entre tiempo de ejecución y presupuesto utilizado.}
\label{fig:grafica_dispersion}
\end{figure}

Los datos sugieren que los jugadores tendieron a sacrificar presupuesto (construyendo túneles costosos) para minimizar el tiempo de viaje, priorizando la velocidad sobre la economía en un [XX]\% de los casos.

\subsection{Resultados de la Encuesta de Percepción}

Tras finalizar la prueba, los participantes evaluaron la usabilidad y el valor educativo del prototipo.

\subsubsection{Usabilidad y Experiencia de Usuario}
Se preguntó sobre la facilidad de uso de las herramientas de construcción. El [XX]\% de los encuestados calificó el sistema de splines como \textit{Intuitivo} o \textit{Muy Intuitivo}. Sin embargo, un [XX]\% reportó dificultades con la cámara en dispositivos con pantallas pequeñas.

\begin{figure}[H]
\centering
% \includegraphics[width=0.8\textwidth]{figures/grafica_encuesta_usabilidad.png} % AGREGAR TU GRAFICA AQUI
\framebox[0.8\textwidth]{\rule{0pt}{5cm} [INSERTAR GRÁFICA DE TORTA: CALIFICACIÓN USABILIDAD]}
\caption{Percepción de la usabilidad del sistema de construcción.}
\label{fig:encuesta_usabilidad}
\end{figure}

\subsubsection{Percepción del Aprendizaje (GBL)}
Respecto al componente educativo, la Figura \ref{fig:encuesta_aprendizaje} refleja la opinión de los estudiantes sobre la adquisición de conocimientos ferroviarios.

\begin{figure}[H]
\centering
% \includegraphics[width=0.8\textwidth]{figures/grafica_encuesta_aprendizaje.png} % AGREGAR TU GRAFICA AQUI
\framebox[0.8\textwidth]{\rule{0pt}{5cm} [INSERTAR GRÁFICA DE BARRAS: TEMAS APRENDIDOS]}
\caption{Temas sobre ingeniería ferroviaria que los usuarios consideran haber aprendido.}
\label{fig:encuesta_aprendizaje}
\end{figure}

Los participantes destacaron principalmente la comprensión de la relación entre la topografía (pendientes) y la capacidad de carga del tren, validando el objetivo pedagógico de la simulación física.

\section{Discusión y Trabajo Futuro}

A partir del análisis de los resultados cuantitativos y cualitativos, se han identificado oportunidades de mejora para futuras iteraciones del proyecto.

\subsection{Errores Detectados}
Durante la prueba piloto de 4 días, se reportaron los siguientes fallos que requieren corrección:
\begin{itemize}
    \item \textbf{Cámara en límites del mapa:} En ocasiones, la cámara permite visualizar el \textit{vacío} si el jugador se desplaza a los bordes extremos del mapa. Se requiere implementar límites rígidos (\textbf{clamping}) en las coordenadas de la cámara.
    \item \textbf{Colisiones fantasma:} En el 2\% de las partidas registradas por la API, el tren detectó daño sin haber una colisión visual clara. Esto sugiere la necesidad de refinar los \textbf{colliders} de las zonas de riesgo.
\end{itemize}

\subsection{Mejoras y Trabajo Futuro}
Para expandir el alcance del videojuego serio, se proponen las siguientes líneas de trabajo:
\begin{enumerate}
    \item \textbf{Expansión del contenido histórico:} Incluir más narrativa sobre la historia real de los Ferrocarriles Nacionales de Colombia, desbloqueable mediante coleccionables en el mapa y mediante dialogos proporcionados por Don Raíl.
    \item \textbf{Expansión del contenido jugable:} Incorporar niveles adicionales con una progresión de dificultad más gradual, de modo que la transición entre etapas sea más equilibrada y se eviten saltos bruscos en el reto para el jugador.
    \item \textbf{Modo "Sandbox":} Habilitar un modo sin restricciones económicas para fomentar la creatividad libre del usuario.
    \item \textbf{Portabilidad WebGL:} Optimizar los recursos gráficos para permitir la ejecución del juego directamente en navegadores web, facilitando su acceso en entornos educativos sin necesidad de instalación.
    \item \textbf{Mejoras de calidad de vida:} Implementar diversas optimizaciones orientadas a la experiencia del usuario, tales como la inclusión de un cronómetro al inicio del recorrido del tren, una mayor visibilidad en las zonas de daño, opciones de configuración de pantalla que permitan elegir entre modo ventana o pantalla completa, así como la incorporación de atajos de teclado para agilizar acciones frecuentes (por ejemplo, utilizar \texttt{Ctrl+Z} para borrar el ultimo tramo creado).
\end{enumerate}