% !TeX root = ../main.tex

\chapter{Resultados}

En este capítulo se presentan los productos resultantes del proceso de desarrollo del prototipo del videojuego serio, abarcando desde la consolidación de los requerimientos y el diseño de la arquitectura, hasta la implementación del prototipo y el análisis de los datos obtenidos durante las pruebas individuales.

\section{Requerimientos}

Los artefactos presentados a continuación son el producto de un proceso de refinamiento continuo llevado a cabo durante las fases de inicio, elaboración e implementación. Tras múltiples ciclos de iteración y ajuste, se obtuvieron estos documentos que constituyen la versión final y consolidada de los requisitos funcionales y el análisis de riesgos que guiaron la implementación del videojuego serio.

\subsection{Matriz de Riesgos}

Para categorizar el estado de las amenazas identificadas, se aplicó la escala de medición definida en la etapa de planificación, en la cual el nivel de riesgo se determina por el producto del Impacto por la Probabilidad de ocurrencia (ver  \cref{fig:escala_riesgos}).
\begin{figure}[H]
\centering
\includegraphics[width=1\textwidth]{figures/linea de impacto-probabilidad.png}
\caption{Escala de evaluación de riesgos utilizada durante el proyecto.}
\label{fig:escala_riesgos}
\end{figure}
 
La cref{tab:matriz-riesgos-final} presenta la matriz de riesgos, donde las columnas representan lo siguiente:

\begin{enumerate}
  \item Ref: Identificador único alfanumérico asignado a cada riesgo para facilitar su seguimiento y referencia en el documento.
  \item Descripción: Detalle narrativo del evento incierto o problema técnico que podría afectar el desarrollo del proyecto.
  \item Causa: Factor, condición o decisión técnica que daría origen a la aparición del riesgo identificado.
  \item Impacto: Estimación de la magnitud de las consecuencias negativas sobre los objetivos del proyecto (rendimiento, tiempo o calidad) en una escala del 1 al 5.
  \item Probabilidad: Calificación numérica de la posibilidad de que el evento de riesgo ocurra durante el ciclo de vida del desarrollo, evaluada del 1 al 5.
  \item Riesgo: Valor cuantitativo obtenido del producto entre el Impacto y la Probabilidad, utilizado para priorizar la criticidad de la amenaza.
  \item Plan de Acción: Conjunto de estrategias, tareas o medidas de mitigación diseñadas para reducir o eliminar el efecto del riesgo en el sistema.
  \item Estado: Se definieron dos posibles estados para cada riesgo: \textit{Abierto}, que indica que el riesgo persiste y requiere de una resolución, y \textit{Cerrado}, que señala que el riesgo se ha materializado, ha sido mitigado exitosamente o se ha aceptado conscientemente que no se realizarán más acciones sobre él.
\end{enumerate}

Tras las iteraciones de desarrollo y las correcciones implementadas en la fase de implementación, todos los riesgos técnicos críticos fueron mitigados y cerrados exitosamente antes de la fase de evaluación.
\pagebreak
\begin{landscapefull} 
\begin{longtable}{|c|L{5cm}|L{5cm}|c|c|c|L{5cm}|c|}
\caption{Matriz de riesgos del proyecto.}\label{tab:matriz-riesgos-final} \\
\hline
\multicolumn{1}{|c|}{\textbf{Ref}} &
\multicolumn{1}{c|}{\textbf{Descripción}} &
\multicolumn{1}{c|}{\textbf{Causa}} &
\multicolumn{1}{c|}{\textbf{Imp.}} &
\multicolumn{1}{c|}{\textbf{Prob.}} &
\multicolumn{1}{c|}{\textbf{Riesgo}} &
\multicolumn{1}{c|}{\textbf{Plan de acción}} &
\multicolumn{1}{c|}{\textbf{Estado}} \\
\hline
\endfirsthead

\hline
\multicolumn{1}{|c|}{\textbf{Ref}} &
\multicolumn{1}{c|}{\textbf{Descripción}} &
\multicolumn{1}{c|}{\textbf{Causa}} &
\multicolumn{1}{c|}{\textbf{Imp.}} &
\multicolumn{1}{c|}{\textbf{Prob.}} &
\multicolumn{1}{c|}{\textbf{Riesgo}} &
\multicolumn{1}{c|}{\textbf{Plan de acción}} &
\multicolumn{1}{c|}{\textbf{Estado}} \\
\hline
\endhead

\hline
\multicolumn{8}{r}{\textit{Continúa en la siguiente página}} \\
\endfoot

\hline
\endlastfoot

R3 & Vías poco naturales e inconexas & Uso de rectas para generar vías & 5 & 4 & 20 & Implementación de curvas Bézier y Splines. & Cerrado \\ \hline

R1 & El modelo del terreno está mal optimizado & Sobrecarga en memoria gráfica por modelo 3D denso & 4 & 4 & 16 & Uso de Terrenos nativos de Unity y carga por bloques (LOD). & Cerrado \\ \hline

R7 & Dificultad para representar hidrografía completa & Densidad excesiva de ríos en Colombia & 4 & 4 & 16 & Selección de ríos principales y lagos mayores únicamente. & Cerrado \\ \hline

R13 & Bajos FPS en equipos de gama baja & Desarrollo en equipos de gama alta & 5 & 3 & 15 & Optimización de draw calls y reducción de polígonos. & Cerrado \\ \hline

R8 & Las vías atraviesan montañas & Falta de interpolación de alturas & 3 & 4 & 12 & Generación automática de túneles al detectar colisión con terreno. & Cerrado \\ \hline

R2 & La cámara atraviesa el terreno & Falta de colisión en la cámara & 2 & 5 & 10 & Implementación de Clamping y colisionadores para la cámara. & Cerrado \\ \hline

R12 & Incompatibilidad de Shaders y URP & Conción por defecto de Unity & 3 & 3 & 9 & Ajuste manual de materiales para el Universal Render Pipeline. & Cerrado \\ \hline

R9 & Estadísticas de locomotoras no leídas & Datos dispersos en hijos del GameObject & 4 & 2 & 8 & Unificar la lógica en el script controlador padre. & Cerrado \\ \hline

R4 & Terreno con parches o vacíos & DEM con datos erróneos & 2 & 3 & 6 & Corrección mediante algoritmos de suavizado. & Cerrado \\ \hline

R6 & Circuito cerrado no detectado & Spline no cerrado & 2 & 3 & 6 & Detección de proximidad de nodos inicial/final. & Cerrado \\ \hline

R10 & HUD no muestra vida tras cambio & Referencias perdidas al cambiar locomotora & 2 & 2 & 4 & Función para reiniciar estadísticas al instanciar. & Cerrado \\ \hline

R11 & Modelos 3D con escalas erróneas & Exportación incorrecta desde Blender & 1 & 3 & 3 & Aplicar escalas y rotaciones antes de exportar. & Cerrado \\ \hline

R5 & Textura de vías solapada & Vía al mismo nivel del suelo & 1 & 2 & 2 & Mover las vías generadas una pequeña distancia hacia arriba. & Cerrado \\ \hline

\end{longtable}
\end{landscapefull} 
\subsection{Definición de Épicas e Historias de Usuario}

Con el objetivo de garantizar que el videojuego cumpliera con los lineamientos del Aprendizaje Basado en Juegos (GBL), el proceso de definición de requisitos comenzó con la elaboración de un conjunto preliminar de Historias de Usuario (HU). Este primer barrido buscó capturar todas las funcionalidades necesarias para cubrir tanto las mecánicas básicas como los objetivos del proyecto.\\

Posteriormente, para facilitar la gestión del desarrollo y la planificación de las iteraciones, estas historias de usuario fueron analizadas y agrupadas en categorías de alto nivel denominadas \textit{Épicas}. Esta organización permitió clasificar los requisitos de manera lógica según el subsistema o área funcional a la que pertenecían.

En la Tabla \cref{tab:epicas} se presentan las 8 épicas resultantes de esta categorización.

\begin{table}[H] 
\centering 
\caption{Listado de Épicas del proyecto.} \label{tab:epicas} 
\begin{tabular}{|c|l|} 
\hline 
\textbf{ID} & \textbf{Nombre de la Épica}\\ \hline 
1 & Selección de Nivel y Contratos \\ \hline 
2 & Interacción con el Mapa 3D \\ \hline 
3 & Sistema de Construcción de Vías \\ \hline 
4 & Gestión de Material Rodante \\ \hline 
5 & Motor de Simulación y Eventos \\ \hline 
6 & Sistema de Evaluación y Puntuación \\ \hline 
7 & Progresión y Rejugabilidad \\ \hline 
8 & Narrativa y Mentoría\\ \hline 
\end{tabular} 
\end{table}

La Tabla \cref{tab:HU-final} presenta el formato consolidado de las historias de usuario tras el proceso de refinamiento continuo tras cada iteración durante las fases del proyecto. Se define las columnas de la siguiente manera:

\begin{enumerate}
    \item Épica: Categoría funcional de alto nivel que agrupa un conjunto de historias de usuario relacionadas con un mismo módulo o subsistema del proyecto definidas anteriormente.
    \item Código: Referencia alfanumérica única asignada a cada requisito (ej. HU1) para facilitar su identificación y rastreabilidad técnica.
    \item Historia de Usuario: Descripción breve de una funcionalidad o necesidad del sistema, redactada desde la perspectiva del rol respectivo.
    \item Prioridad: Nivel de importancia asignado mediante el método MoSCoW, clasificando las tareas en: \textit{Must have} (M) como imprescindibles, \textit{Should have} (S) como importantes, \textit{Could have} (C) como deseables y \textit{Won't have} (W) para futuras implementaciones.
    \item Complejidad: Valor numérico en una escala de 1 a 5 que estima el grado de dificultad técnica y el esfuerzo requerido para la implementación de la funcionalidad.
\end{enumerate}
\pagebreak
\begin{landscapefull} 
\begin{longtable}{|c|c|L{15cm}|c|c|}
\caption{Tabla final de historias de usuario implementadas.} \label{tab:HU-final} \\
\hline
\multicolumn{1}{|c|}{\textbf{ÉPICA}} &
\multicolumn{1}{c|}{\textbf{CÓDIGO}} &
\multicolumn{1}{c|}{\textbf{HISTORIA DE USUARIO}} &
\multicolumn{1}{c|}{\textbf{PRIO.}} &
\multicolumn{1}{c|}{\textbf{COMP.}} \\ \hline
\endfirsthead

\hline
\multicolumn{1}{|c|}{\textbf{ÉPICA}} &
\multicolumn{1}{c|}{\textbf{CÓDIGO}} &
\multicolumn{1}{c|}{\textbf{HISTORIA DE USUARIO}} &
\multicolumn{1}{c|}{\textbf{PRIO.}} &
\multicolumn{1}{c|}{\textbf{COMP.}} \\ \hline
\endhead

\hline
\multicolumn{5}{r}{\textit{Continúa en la siguiente página}} \\
\endfoot

\hline
\endlastfoot

1 & HU1  & Como jugador, quiero ver una pantalla de selección de niveles para elegir cuál jugar. & M & 2 \\ \hline
1 & HU2  & Como jugador, quiero que al empezar un nivel, se me presente una pantalla de contrato con presupuesto y objetivos. & M & 1 \\ \hline
2 & HU3  & Como jugador, quiero poder mover la cámara (panorámica) por el mapa para explorar el terreno. & M & 1 \\ \hline
2 & HU4  & Como jugador, quiero poder acercar y alejar la vista (zoom) para ver el mapa con diferentes niveles de detalle. & M & 1 \\ \hline
2 & HU5  & Como jugador, quiero ver las delimitaciones políticas y ríos en el mapa. & M & 3 \\ \hline
2 & HU6  & Como jugador, quiero ver un terreno en 3D del nivel a jugar. & M & 4 \\ \hline
2 & HU7  & Como jugador, quiero poder rotar la cámara para observar el terreno desde distintos ángulos. & S & 1 \\ \hline
2 & HU8  & Como jugador, quiero poder activar una capa de vista topográfica para entender el relieve. & C & 5 \\ \hline
3 & HU10 & Como jugador, quiero poder construir un tramo de vía férrea entre dos puntos. & M & 2 \\ \hline
3 & HU11 & Como jugador, quiero poder eliminar un tramo de vía férrea para rediseñar mis rutas. & M & 1 \\ \hline
3 & HU12 & Como jugador, quiero que se construyan puentes automáticamente al cruzar valles. & M & 4 \\ \hline
3 & HU13 & Como jugador, quiero que se construya automáticamente un túnel si mi vía férrea atraviesa una montaña. & M & 4 \\ \hline
3 & HU14 & Como jugador, quiero poder construir estaciones para definir los puntos de inicio y fin. & M & 3 \\ \hline
3 & HU15 & Como jugador, quiero ver mi presupuesto actual en todo momento. & M & 2 \\ \hline
3 & HU16 & Como jugador, quiero poder deshacer mi última acción de construcción. & S & 2 \\ \hline
4 & HU17 & Como jugador, quiero acceder a un menú para seleccionar y comprar la locomotora. & M & 3 \\ \hline
4 & HU18 & Como jugador, quiero añadir vagones específicos (pasajeros, carga) a mi locomotora. & M & 2 \\ \hline
4 & HU19 & Como jugador, quiero que el menú me muestre solo las locomotoras aptas para la ruta diseñada. & C & 2 \\ \hline
5 & HU20 & Como jugador, quiero ver una alerta visual cuando el tren sufre daños en zonas de evento. & M & 2 \\ \hline
5 & HU21 & Como jugador, quiero que el botón de "Testeo" me informe si falta algún requisito. & S & 3 \\ \hline
5 & HU22 & Como jugador, quiero volver a la construcción después de un testeo para corregir la ruta. & S & 1 \\ \hline
6 & HU23 & Como jugador, al finalizar un nivel, quiero ver una pantalla de puntuación. & M & 3 \\ \hline
6 & HU24 & Como jugador, quiero que la puntuación considere la rentabilidad (costo vs. contrato) y durabilidad. & M & 3 \\ \hline
6 & HU25 & Como jugador, quiero que la puntuación considere la eficiencia (tiempo empleado). & S & 3 \\ \hline
6 & HU26 & Como jugador, quiero ver un informe detallado que desglose mi puntuación final. & S & 3 \\ \hline
7 & HU27 & Como jugador, quiero desbloquear nuevos niveles tras completar el contrato actual. & M & 2 \\ \hline
8 & HU28 & Como jugador, quiero que Don Raíl presente el contrato con un diálogo breve y contexto histórico. & S & 3 \\ \hline
8 & HU29 & Como nuevo jugador, quiero que Don Raíl me guíe con un tutorial interactivo en el primer nivel. & S & 4 \\ \hline
8 & HU30 & Como jugador, quiero recibir advertencias contextuales sobre pendientes pronunciadas. & S & 4 \\ \hline
8 & HU31 & Como jugador, quiero ver datos históricos al seleccionar una ciudad importante. & S & 4 \\ \hline
8 & HU32 & Como jugador, quiero comentarios de Don Raíl cuando el tren sufre un evento. & S & 3 \\ \hline
8 & HU33 & Como jugador, quiero que Don Raíl comente mi puntuación final (éxito o consejo). & S & 2 \\ \hline
8 & HU34 & Como jugador, quiero ver anécdotas ferroviarias en las pantallas de carga. & M & 2 \\ \hline
6 & HU35 & Como administrador, quiero guardar metadatos de los jugadores en una base de datos. & M & 3 \\ \hline

\end{longtable}
\end{landscapefull} 

\section{Diseño}

El diseño de la arquitectura de software un videojuego serio, se orientó a satisfacer los requisitos fundamentales del marco GBL, priorizando la articulación coherente de las HU y la mitigación efectiva de los riesgos técnicos identificados previamente. Más allá de la estabilidad funcional, esta estructura se concibió para soportar una integración transversal con servicios externos, garantizando que la simulación ferroviaria gestione eficientemente el intercambio de datos vía API para la telemetría del jugador y la generación posterior de informes de auditoría.

\subsection{Arquitectura de Escenas en Unity}

Para optimizar la gestión de recursos y organizar lógicamente los componentes del videojuego, se implementó una arquitectura modular basada en escenas aditivas. A diferencia de los entornos monolíticos tradicionales, este sistema permite fragmentar la aplicación en capas independientes que se gestionan dinámicamente en tiempo de ejecución.

Como se ilustra en la \cref{fig:escenas_aditivas}, el diagrama describe el ciclo de vida de estas capas a través de tres zonas funcionales. En la parte izquierda (zona de carga), el sistema invoca e instancia nuevas escenas de manera aditiva, sumándolas al entorno sin detener la ejecución actual. Estas pasan a situarse en la sección central (zona de apilamiento), donde coexisten y operan simultáneamente como capas superpuestas, manteniendo su independencia lógica aunque compartan el mismo espacio de memoria.

Finalmente, el flujo concluye en la parte derecha (zona de descarga), la cual representa la liberación selectiva de recursos. En esta etapa, el sistema puede retirar una capa específica de la pila —como un nivel completado— sin afectar a las demás escenas persistentes. Esta estrategia de "quitar y poner" capas evita la reinicialización redundante de sistemas transversales como la del terreno de juego, garantizando transiciones fluidas y un uso eficiente del procesador.

\begin{figure}[H]
\centering
\includegraphics[width=0.9\textwidth]{figures/escenas_aditivas.png}
\caption{Diagrama de la arquitectura basada en escenas aditivas.}
\label{fig:escenas_aditivas}
\end{figure}

En nuestro caso, el flujo de ejecución comienza con la carga simultánea de las escenas \textit{MainMenu} y \textit{MapaMenu}. Posteriormente, al iniciar la partida, se descarga la escena decorativa y se carga la escena \textit{MapaCompleto} junto con la escena del nivel correspondiente (por ejemplo, \textit{Nivel 1 Tutorial}), manteniendo siempre activa la escena base de gestión.

A continuación, se describen las 8 escenas que componen el videojuego serio dentro de Unity:

\begin{enumerate}
    \item MainMenu: Es la escena persistente que contiene el lienzo (\textit{Canvas}) principal de la interfaz y los gestores globales del sistema. Los objetos instanciados aquí están condos para no destruirse entre cargas (\textit{DontDestroyOnLoad}), asegurando la continuidad de la música, los datos del jugador y la gestión de estados.
    
    \item MapaMenu: Funciona como una escena puramente decorativa para el fondo del menú principal. Consiste en un fragmento extraído del mapa completo (exactamente 1 \textit{tile} de terreno) en el cual se ejecutan animaciones de trenes en bucle, brindando contexto visual sin cargar la totalidad de la geografía.
    
    \item MapaCompleto: Es la escena más compleja y pesada del sistema, encargada de contener la totalidad de la representación geográfica. Esta escena integra la malla del terreno dividida en una cuadrícula de $16 \times 16$ \textit{tiles}, la red hidrográfica completa (ríos lineales y cuerpos de agua), las delimitaciones políticas administrativas y el etiquetado geográfico de todos los municipios posicionados sobre la topografía.
    
    \item Niveles (1 al 5): Corresponden a 5 escenas independientes (\textit{Nivel 1 Tutorial}, \textit{Nivel 2}, etc.) que se cargan de forma aditiva sobre el mapa completo. Estas escenas contienen únicamente los objetos necesarios para la jugabilidad específica de ese contrato: los puntos de inicio y fin, la conción de las mecánicas, los gestores de eventos locales y los \textit{Canvas} específicos de la misión.
\end{enumerate}

\subsection{Diagramas de Proceso}

En el diagrama que se presenta en la \cref{fig:diagramaUsoMenu}, se muestra la forma en la que la interacción con el menú principal es realizada por parte del jugador, siguiendo todos los casos de cada interacción por parte del jugador con todos los elementos de la interfaz del juego, además, notese como el sistema, una vez detectado la falta de un archivo de guardado, se encargará de generar uno y activar los tutoriales para que el jugador pueda comenzar.

\begin{figure}[htbp]
\centering
\includegraphics[width=0.9\textwidth]{figures/diagramaUsoMenu.drawio.png}
\caption{Diagrama de proceso del menú.}
\label{fig:diagramaUsoMenu}
\end{figure}

Por su parte, en el diagrama que se presenta en la \cref{fig:diagramaUsonivela} y \cref{fig:diagramaUsonivelb} se demuestra la forma en la que una partida es realizada por el jugador, tomando en cuenta la interacción con cada elemento de la interfaz y con los elementos físicos como los son el mouse (clicls y arrastre) y el teclado (teclas WASD). De manera general, el proceso abarca desde la aceptación del contrato y la gestion de movimiento del jugador, pasando por la construcción de vías y la elección de los trenes a utilizar, hasta la ejecución de la simulación, donde se evalúan los criterios de victoria o derrota. Finalmente, el sistema automatiza el envío de métricas a la API y la persistencia de datos en la base de datos NoSQL.

\begin{figure}[htbp]
\centering
\includegraphics[
            width=\linewidth,    % Ocupa todo el ancho disponible
            height=1\textheight, % Ocupa máximo el 85% del alto (para dejar espacio al caption)
            keepaspectratio, trim=0 0 650 0, clip]{figures/diagramaprocenivela.drawio.png}
\caption{Diagrama de proceso del menú (Figura a).}
\label{fig:diagramaUsonivela}
\end{figure}

\begin{figure}[htbp]
\centering
\includegraphics[width=0.9\textwidth]{figures/diagramaprocenivelb.drawio.png}
\caption{Diagrama de proceso del menú (Figura b).}
\label{fig:diagramaUsonivelb}
\end{figure}

\subsection{Diagrama de Componentes}

Dada la división en múltiples escenas y el diseño detallado de los scripts e iteraciones del sistema, se optó por representar esta estructura en un diagrama de componentes. Con el fin de facilitar la lectura, dicho diagrama se fragmentó en módulos que corresponden a cada escena como lo son la gestión del mapa, el diseño de niveles y la interfaz de usuario. Más adelante se presentará la descripción a detalle de cada escena junto con sus componentes principales.

\afterpage{
\begin{landscapefull} 
    \begin{figure}[H]
        \centering
        \includegraphics[
            width=\linewidth,    % Ocupa todo el ancho disponible
            height=0.85\textheight, % Ocupa máximo el 85% del alto (para dejar espacio al caption)
            keepaspectratio      % VITAL: Evita que la imagen se deforme al estirarse
        ]{figures/test_DiagramaComponentes.png}
        \caption{Diagrama de componentes organizado por escenas de Unity.}
        \label{fig:diagrama_componentes}
    \end{figure}
\end{landscapefull}}

\subsubsection{Main Menu}

El Main Menu es la escena encargada de mostrar la interfaz inicial que verá el jugador al abrir el juego serio, desde esta escena es que se podrán acceder a los niveles y las Conciones.

En esta escena podemos encontrar componentes como:

\begin{enumerate}
  \item SaveSystem: Es el encargado de manejar el archivo de guardado del progreso del jugador.
  \item System\_SceneLoader: Se encarga de manejar la carga de las distintas escenas que utilizarán.
  \item TutorialManager: Se encarga de los diálogos y controles durante el tutorial la primera vez que se ejecuta el juego.
  \item EventSystem: Un componente nativo de unity usado para la deteccion de interacciones.
  \item MúsicaSonido y UISonido: Encargados del manejo del sonido tanto musical como de interacción.
  \item Canvas: componente que se encarga de todo lo visual y de interacción en el menú, como son los botones, sliders, etc.
\end{enumerate}

\afterpage{
\begin{landscapefull}
\begin{figure}[H] % 'p' sugiere que ocupe una página completa
\centering
\includegraphics[
            width=\linewidth,    % Ocupa todo el ancho disponible
            height=0.85\textheight, % Ocupa máximo el 85% del alto (para dejar espacio al caption)
            keepaspectratio      % VITAL: Evita que la imagen se deforme al estirarse
        ]{figures/mainmenu.png} % Nota: width basado en textheight porque la página está rotada
\caption{Diagrama de componentes del Menú Principal}
\label{fig:diagrama_mainmenu}
\end{figure}
\end{landscapefull}}

\subsubsection{Mapa Menu}

La escena de Mapa Menu es la encargada de mostrar el mapa que se ve de fondo cuando el jugador esta en el menú principal, conteniendo tanto el mapa como los trenes en movimiento.

En esta escena podemos encontrar componentes como:

\begin{enumerate}
    \item Terreno\_ColombiaV2: Contenedor del fragmento del mapa que vemos en el menu.
    \item Agua: Se encarga de almacenar los ríos.
    \item PathSplineMenu: Se encarga de gestionar los spline de las vías por las que los trenes transitan.
    \item Tren1, Tren2 y Tren3: Son las 3 variantes de trenes junto a sus respectivos vagones.
    \item Infraestructura: Contiene los puentes y tuneles por los que vemos a los trenes transitar.
    \item Bandera: El simbolo patrio, contiene su modelo y su animacion de ser meneada por el viento.
\end{enumerate}

\afterpage{
\begin{landscapefull}
\begin{figure}[H] % 'p' sugiere que ocupe una página completa
\centering
\includegraphics[
            width=\linewidth,    % Ocupa todo el ancho disponible
            height=0.85\textheight, % Ocupa máximo el 85% del alto (para dejar espacio al caption)
            keepaspectratio      % VITAL: Evita que la imagen se deforme al estirarse
        ]{figures/mapamenu.png} 
\caption{Diagrama de componentes del Mapa del Menu}
\label{fig:diagrama_mapamenu}
\end{figure}
\end{landscapefull}}

\subsubsection{MapaCompleto}

La escena del Mapa completo se encarga de mostrar el mapa completo de todo el pais, conteniendo tanto el mapa como los cuerpos de agua.

En esta escena podemos encontrar componentes como:

\begin{enumerate}
    \item Terreno\_ColombiaV2: Contenedor del mapa completo del país, contando con 256 fragmentos del mapa que conectados forman toda la extension territorial de Colombia.
    \item Agua: Se encarga de almacenar los ríos, lagos y otros cuerpos de agua, contando con mas de 40,000 cuerpos de agua almacenados.
    \item Fronteras Municipales: En este estan contenidas todas las fronteras de los municipios.
    \item Fronteras Departamentales: En este estan contenidas todas las fronteras de los departamentos.
    \item Fronteras Nacional: En este estan contenidas las fronteras nacionales con otros paises como lo son Panamá, Peru, Brasil, Venezuela, y Ecuador.
    \item Etiquetas Municipios: Contiene las etiquetas de todos los municipios de Colombia, teniendo etiquetas desde Abejorral hasta Zipaquira.
    \item Etiquetas Departamentales: Contiene las etiquetas de todos los departamentos de Colombia, teniendo etiquetas desde Amazonas hasta Vichada.
    \item PlanoOceanoNoContruye: Un plano 3D gigante, con textura transparente, que cubre el area de todo el pais, asignado a la altura 0, es usado como freno para que los jugadores no construyan en el oceano.
    \item LevelLoader: Este elemento esta encargado de manejar la carga de diferentes escenas.
\end{enumerate}

\afterpage{
\begin{landscapefull}
\begin{figure}[H] % 'p' sugiere que ocupe una página completa
\centering
\includegraphics[
            width=\linewidth,    % Ocupa todo el ancho disponible
            height=0.85\textheight, % Ocupa máximo el 85% del alto (para dejar espacio al caption)
            keepaspectratio      % VITAL: Evita que la imagen se deforme al estirarse
        ]{figures/mapacompleto.png} 
\caption{Diagrama de componentes del Mapa completo}
\label{fig:diagrama_mapacompleto}
\end{figure}
\end{landscapefull}}

\subsubsection{Nivel 1 Tutorial}

En esta escena se encuentran objetos tanto de visibilidad del jugador como de analisis y calculos de resultados, asi como también manejar el tutorial para guiar al jugador durante su primera experiencia con el videojuego serio

\begin{enumerate}
    \item Jugador: En este item se encuentra todo lo referente al jugador, tales como el controlador de movimiento, la cámara y las funciones de deteccion y construccion de vías.
    \begin{enumerate}
        \item Controlador de movimiento: Encargado de recibir y ejecutar los movimientos que el jugador haga por medio del mouse y el teclado
        \item Cámaras: Encargadas de ser los ojos del jugador, también cuenta con una cámara secundaria que actua como vista aerea encima del país
        \item Splines: Este componente contiene el script encargado de la creacion de las vías férreas, la previsualizacion de construccion y la construccion de estructuras como tuneles y puentes.
        \end{enumerate}

    \item Terreno: En este componente se encuentran almacenados los elementos unicos referentes al nivel, tales como lo son las zonas de daño y los puntos por los que debe pasar el tren.
    \item Canvas Nivel: En este canvas estan contenidos todos los elementos de interfaz con los que puede interactuar el jugador, tales como botones y sliders.
    \begin{enumerate}
        \item Nivel: Este componente contiene todos los elementos que el jugador puede ver durante el desarrollo normal del nivel, entre estos elementos se encuentran las interfaces de vida y dinero, los botones para la activación de modo de construccion y la selección de trenes.
        \item Pausa: Encargado de almacenar los componentes del menú de pausa, tales como los botones de reinicio, guia y volver al menú principal.
        \item Fin de Nivel: Aqui se encuentran las interfaces de victoria y derrota.
        \item Tutorial: Encargado de almacenar los elementos de la interfaz del tutorial, como lo son los cuadros de dialogo y los sprites de "Don Raíl".
        \item ContrataciónLvl: Encargado de almacenar los elementos del contrato del nivel, como los detalles de paradas y dinero disponible.
        \end{enumerate}
    \item Tren: Este componente contiene los objetos de las distintas locomotoras y vagones a los que se tienen acceso durante el desarrollo de los niveles.
    \item Managers: Aqui esta almacenada la lógica interna y de gestión del nivel, cuenta con los siguientes componentes
    \begin{enumerate}
        \item SplineLevel: Este componente es el corazón del nivel, en él se encuentran los parametros necesarios para gestionar el nivel, como lo son las estaciones por las que hay que pasar, las zonas de daño que se encontrarán en el nivel y es el encargado de decidir si un intento cuenta como una victoria o una derrota.
        \item Map Manager: Encargado de almacenar los iconos de los distintos puntos del mapa, como lo son los puertos, las ciudades e incluso el jugador.
        \item AudioManager: Encargado de la gestión de sonido del juego, comprende elementos como la musica y los efectos de sonido del tren en movimiento.
        \item TutorialManager: En él se encuentran almacenados los diálogos y acciones que se espera que el jugador realice durante el desarrollo del tutorial
        \item EconomiaManager: Encargado de la gestión económica del nivel, se encarga de analizar los gastos y gestionar los calculos monetarios del nivel.
    \end{enumerate}
\end{enumerate}

\begin{landscapefull}
\begin{figure}[H] 
\centering
\includegraphics[width=\linewidth,height=0.85\textheight,keepaspectratio,trim=0 0 2350 0,clip]{figures/niveltutorial.png} 
\caption{Diagrama de componentes del Nivel 1 Tutorial (Figura a)}
\label{fig:diagrama_niveltutoriala}
\end{figure}
\end{landscapefull}

\begin{landscapefull}
\begin{figure}[H] 
\centering
\includegraphics[width=\linewidth,height=0.85\textheight,keepaspectratio,trim=1400 0 0 0,clip]{figures/niveltutorial.png} 
\caption{Diagrama de componentes del Nivel 1 Tutorial (Figura b)}
\label{fig:diagrama_niveltutorialb}
\end{figure}
\end{landscapefull}
\subsubsection{Niveles 2 a 5}

Similar a la escena 1 tutorial, los niveles 2 a 5 comparten una estructura similar, los cambios más destacables son la remoción del tutorial y cambios individuales a cada nivel para determinar los objetivos del mismo

\begin{enumerate}
    \item Jugador: En este item se encuentra todo lo referente al jugador, tales como el controlador de movimiento, la cámara y las funciones de deteccion y construccion de vías.
    \begin{enumerate}
        \item Controlador de movimiento: Encargado de recibir y ejecutar los movimientos que el jugador haga por medio del mouse y el teclado
        \item Cámaras: Encargadas de ser los ojos del jugador, también cuenta con una cámara secundaria que actua como vista aerea encima del país
        \item Splines: Este componente contiene el script encargado de la creacion de las vías férreas, la previsualizacion de construccion y la construccion de estructuras como tuneles y puentes.
        \end{enumerate}

    \item Terreno: En este componente se encuentran almacenados los elementos unicos referentes al nivel, tales como lo son las zonas de daño y los puntos por los que debe pasar el tren.
    \item Canvas Nivel: En este canvas estan contenidos todos los elementos de interfaz con los que puede interactuar el jugador, tales como botones y sliders.
    \begin{enumerate}
        \item Nivel: Este componente contiene todos los elementos que el jugador puede ver durante el desarrollo normal del nivel, entre estos elementos se encuentran las interfaces de vida y dinero, los botones para la activación de modo de construccion y la selección de trenes.
        \item Pausa: Encargado de almacenar los componentes del menu de pausa, tales como los botones de reinicio, guia y volver al menú principal.
        \item Fin de Nivel: Aqui se encuentran las interfaces de victoria y derrota.
        \item ContrataciónLvl: Encargado de almacenar los elementos del contrato del nivel, como los detalles de paradas y dinero disponible.
        \end{enumerate}
    \item Tren: Este componente contiene los objetos de las distintas locomotoras y vagones a los que se tienen acceso durante el desarrollo de los niveles.
    \item Managers: Aqui esta almacenada la lógica interna y de gestión del nivel, cuenta con los siguientes componentes
    \begin{enumerate}
        \item SplineLevel: Este componente es el corazón del nivel, en él se encuentran los parametros necesarios para gestionar el nivel, como lo son las estaciones por las que hay que pasar, las zonas de daño que se encontrarán en el nivel y es el encargado de decidir si un intento cuenta como una victoria o una derrota.
        \item Map Manager: Encargado de almacenar los iconos de los distintos puntos del mapa, como lo son los puertos, las ciudades e incluso el jugador.
        \item AudioManager: Encargado de la gestión de sonido del juego, comprende elementos como la musica y los efectos de sonido del tren en movimiento.
        \item EconomiaManager: Encargado de la gestión económica del nivel, se encarga de analizar los gastos y gestionar los calculos monetarios del nivel.
    \end{enumerate}
\end{enumerate}

\begin{landscapefull}
\begin{figure}[H] % 'p' sugiere que ocupe una página completa
\centering
\includegraphics[
            width=\linewidth,    % Ocupa todo el ancho disponible
            height=0.85\textheight, % Ocupa máximo el 85% del alto (para dejar espacio al caption)
            keepaspectratio, trim=0 0 2200 0, clip
        ]{figures/niveles25.png} 
\caption{Diagrama de componentes de los niveles 2 a 5 (Figura a).}
\label{fig:diagrama_niveles25a}
\end{figure}
\end{landscapefull}

\begin{landscapefull}
\begin{figure}[H] % 'p' sugiere que ocupe una página completa
\centering
\includegraphics[
            width=\linewidth,    % Ocupa todo el ancho disponible
            height=0.85\textheight, % Ocupa máximo el 85% del alto (para dejar espacio al caption)
            keepaspectratio, trim=1450 0 0 0, clip
        ]{figures/niveles25.png} 
\caption{Diagrama de componentes de los niveles 2 a 5 (Figura b).}
\label{fig:diagrama_niveles25b}
\end{figure}
\end{landscapefull}

\subsection{Arquitectura del Sistema y API de Informes}

Con el propósito de fortalecer el componente de \textit{Apoyo al Aprendizaje}, requisito fundamental en el diseño GBL, y trascender el sistema básico de puntuación numérica y por estrellas, se implementó una arquitectura Cliente-Servidor para la gestión de datos telemétricos. En este esquema, el cliente (videojuego en Unity) se comunica con un servidor \textit{backend} responsable de procesar y almacenar la actividad de las sesiones. Esto faculta al administrador para analizar los resultados y potenciar la realimentación hacia el jugador mediante la generación de informes detallados.

La  \cref{fig:arquitectura_api} ilustra el flujo de información:

\begin{figure}[H]
\centering
\includegraphics[width=0.9\textwidth]{figures/diseño_preliminar.png} 
\caption{Arquitectura del sistema e integración con la API de informes.}
\label{fig:arquitectura_api}
\end{figure}

\begin{enumerate}
    \item Cliente Unity (C\#): Al finalizar un nivel, el cliente recopila métricas clave y empaqueta estos datos en un objeto JSON.
    \item API REST: Se desarrolló una API intermedia que recibe las peticiones POST provenientes del videojuego. Esta API valida la integridad de los datos y gestiona la seguridad de las transacciones.
    \item Base de Datos: Los registros se almacenan en una base de datos No-SQL (FireBase), permitiendo la persistencia histórica de los intentos de los jugadores.
    \item Módulo de Generación de Informes: Un servicio independiente consulta la base de datos para generar visualizaciones estadísticas y reportes de rendimiento, los cuales son utilizados para la evaluación pedagógica presentada más adelante en este capítulo.
\end{enumerate}

\subsubsection{Arquitectura de datos}
La razón en usar una base de datos No-SQL vs SQL se debe a la naturaleza de la arquitectura de datos en el desarrollo de un videojuego; durante sus ciclos de vida un videojuego a comparación de otros productos software, sufren muchos cambios a nivel de estructura de datos. Por ejemplo, grandes empresas de videojuegos como \textit{RiotGames} o \textit{Electronics Arts} hoy en día utilizan DynamoDB, una base de datos No-SQL para sus videojuegos \autocite{AmazonDynamoDB}.
Para garantizar la persistencia y el análisis posterior de la telemetría, se diseñó un esquema de datos documental alojado en la infraestructura de Firebase. La  \cref{fig:bd_nosql} ilustra la estructura de la colección principal denominada \textit{level\_logs}, la cual actúa como un registro histórico de cada sesión de juego de un nivel terminada por el usuario. Este modelo utiliza una relación de composición para almacenar los datos de interacción espacial directamente dentro del documento de la sesión, optimizando así las consultas de lectura.

\begin{figure}[H]
\centering
\includegraphics[width=0.9\textwidth]{figures/uml_bd_nosql.png} 
\caption{Colección única de la base de datos FireBase en la API.}
\label{fig:bd_nosql}
\end{figure}

A continuación, se detalla la función de cada atributo definido en el esquema:

\begin{enumerate}
    \item sessionID (string): Identificador único generado por la API para diferenciar cada intento de juego, permitiendo trazabilidad individual incluso si un mismo jugador repite el nivel múltiples veces.
    
    \item playerID (string): Referencia al identificador del usuario autenticado. Funciona como una clave foránea lógica que vincula la sesión con los metadatos del perfil del jugador.
    
    \item levelNumber (integer): Indica el nivel específico que se está jugando (ej. 1 a 5), contextualizando la dificultad y los objetivos activos durante la sesión.
    
    \item status (string): Describe el estado final de la partida. Puede tomar valores categóricos como \textit{Victoria}, \textit{Derrota} o \textit{Abandonado}, siendo fundamental para calcular tasas de éxito.
    
    \item causeOfFailure (string): Campo opcional que almacena la razón específica de la derrota (ej. \textit{Descarrilamiento}, \textit{Presupuesto Agotado}). Si el estado es de victoria, este campo permanece nulo o vacío.
    
    \item budgetUsed (float): Variable económica que registra la cantidad total de dinero virtual gastado por el jugador en la construcción de vías y compra de maquinaria.
    
    \item score (integer): Puntuación final obtenida basada en la eficiencia, economía y cumplimiento del contrato del nivel.
    
    \item durationSeconds (float): Tiempo total de juego desde el inicio del nivel hasta la finalización o fallo, utilizado para medir la curva de aprendizaje.
    
    \item trainTravelTime (float): Métrica específica que contabiliza únicamente el tiempo que el tren estuvo en movimiento sobre las vías, separando el tiempo de construcción del tiempo de simulación.
    
    \item timestamp (time): Marca temporal exacta del servidor al momento de guardar el registro, esencial para el ordenamiento cronológico de los datos.
    
    \item clickData (Array de Objetos): Estructura anidada (sub-documento) que almacena una lista de coordenadas vectoriales. Cada elemento contiene:
    \begin{enumerate}
        \item x (float): Posición horizontal del clic en el mundo del juego.
        \item y (float): Posición vertical del clic.
    \end{enumerate}
\end{enumerate}

\section{Implementación}
La fase de implementación se llevó a cabo utilizando el motor Unity (versión 6.0), haciendo uso de Git (ver  \cref{fig:interfaz_git}) para el control de versiones y LFS para la gestión de activos pesados. A continuación, se detalla el proceso técnico de construcción del entorno y las mecánicas ferroviarias.

\begin{figure}[H]
\centering
\includegraphics[width=1\textwidth]{figures/interfaz_git.png}
\caption{Interfaz del repositorio git}
\label{fig:interfaz_git}
\end{figure}

\subsection{Conción del Espacio de Trabajo en Unity}

Para el desarrollo del proyecto se configuró un entorno de trabajo optimizado en Unity 6.0, diseñado para gestionar la complejidad del mapa de Colombia y la lógica de los sistemas ferroviarios. En la  \cref{fig:interfaz_unity} se observa la disposición de las herramientas y la interacción entre las ventanas del editor:

\begin{enumerate}
    \item Jerarquía (Hierarchy): En esta sección se evidencia el uso de un flujo de trabajo basado en múltiples escenas cargadas simultáneamente (\textit{Multi-Scene Editing}), como \textit{MapaCompleto}, \textit{Nivel 1} y \textit{Nivel 2}. Cada escena agrupa sus propios componentes y lógica de forma independiente. En este caso, se encuentra desplegado el \textit{Nivel 2}, donde se ha seleccionado el objeto \textit{Locomotora}, permitiendo una edición focalizada sin alterar el resto del entorno global.
    
    \item Vista de Escena y Juego (Scene \& Game View): El área central permite alternar entre la edición y la visualización final. La pestaña activa (\textit{Scene}) muestra el espacio tridimensional donde se realiza el ensamblaje del relieve y la ubicación de objetos. Adyacente a esta, se encuentra la pestaña \textit{Game}, la cual permite previsualizar la cámara del jugador y la interfaz de usuario final para realizar pruebas de ejecución inmediatas.
    
    \item Inspector: Esta ventana actúa de forma reactiva a la selección realizada en la Jerarquía. Al estar señalada la \textit{Locomotora}, el Inspector despliega todos los componentes asignados a dicho objeto. Desde aquí se gestionan propiedades como el \textit{Transform}, el sistema de audio y las variables físicas del motor del tren, permitiendo ajustar el comportamiento del vehículo en tiempo real.
    
    \item Explorador de Proyecto (Project Window): Contiene la biblioteca de archivos del juego organizada mediante una estructura de carpetas estandarizada (\textit{Meshes, Prefabs, Scripts, Settings}). Esta organización es fundamental para el control de versiones, asegurando que todos los activos del mapa de Colombia y los modelos ferroviarios estén correctamente indexados.
\end{enumerate}

\begin{figure}[H]
\centering
\includegraphics[width=1\textwidth]{figures/interfaz_unity.jpeg}
\caption{Interfaz de desarrollo en Unity}
\label{fig:interfaz_unity}
\end{figure}

\subsection{Construcción del Entorno Geográfico}

El mapa de Colombia representa el tablero de juego principal. Su implementación pasó por múltiples iteraciones para equilibrar el realismo geográfico con el rendimiento técnico.

Inicialmente, se importó un modelo 3D de alta resolución generado en Blender a partir de mapas de elevación (DEM). Como se observa en la  \cref{fig:mapa_inicial}, aunque visualmente detallado, este enfoque resultó inviable debido a la carga de 24.5 millones de vértices, lo que redujo la tasa de refresco a menos de 26 FPS (Fotogramas por segundo).

\begin{figure}[H]
\centering
\includegraphics[width=0.8\textwidth]{figures/mapa_colombia_v1.jpeg}
\caption{Primer intento de generar el terreno usando un modelo 3D en Blender.}
\label{fig:mapa_inicial}
\end{figure}

Tras experimentar con versiones de baja cantidad de polígonos (\textit{low-poly}) y fragmentadas que no solucionaron los problemas de latencia en las llamadas de renderizado(\textit{draw-calls}), se optó por la solución de usar el sistema de terrenos nativo de Unity.

\subsubsection{Automatización y Procesamiento de Datos Geográficos}

Para la implementación definitiva, no se esculpió el terreno manualmente. En su lugar, se desarrolló un flujo de trabajo automatizado mediante \textit{scripts} en C\# que generaban la topografía a partir de datos reales.

Una herramienta fundamental en esta etapa fue el software QGIS ( Quantum Geographic Information System). Esta herramienta permitió procesar y alinear todas las capas de información —fronteras municipales, hidrografía (ríos y lagos) y elevación— bajo un Sistema de Referencia de Coordenadas unificado, específicamente el \textit{EPSG:9377} (MAGNA-SIRGAS Origen Nacional). Esto garantizó que todos los elementos geográficos mantuvieran su posición relativa correcta al ser importados al motor de videojuego. En la  \cref{fig:qgis_1} se muestra todos los datos recopilados y correctamente ubicados como son el mapa de alturas, los cuerpos de agua, las fronteras nacional, departamentales y municipales de Colombia.

\begin{figure}[H]
\centering
\includegraphics[width=1\textwidth]{figures/qgis_obramaestra.png}
\caption{Captura del proyecto en QGIS utilizado para el terreno.}
\label{fig:qgis_1}
\end{figure}

El Modelo Digital de Elevación (DEM) utilizado como base fue obtenido a través del \textit{General Bathymetric Chart of the Oceans} \autocite{GEBCO}. Dada la extensión del territorio, se decidió segmentar este DEM en una cuadrícula de $16 \times 16$ secciones, generando terrenos interconectados dentro de Unity a una escala de 1:128.

\subsubsection{Estrategia de Segmentación (Chunks)}

Esta división en 256 \textit{tiles} o baldosas respondía a una estrategia inicial de optimización basada en la carga dinámica de \textit{chunks}. En este contexto, un \textit{chunk} se definió como una unidad lógica compuesta por un fragmento de terreno y la totalidad de los objetos (etiquetas, cuerpos de agua, divisiones políticas) situados sobre él. La intención original era cargar y descargar estos bloques de memoria en tiempo real según la posición del jugador.

\begin{figure}[H]
\centering
\includegraphics[width=0.8\textwidth]{figures/codigo_importador_terrenos.png}
\caption{Fragmento del script encargado de la importación de terrenos.}
\label{fig:cod_1}
\end{figure}

Sin embargo, durante las pruebas de integración, se observó que el rendimiento del sistema era estable incluso con la totalidad del mapa cargado en memoria. Por consiguiente, se descartó el sistema de carga dinámica para reducir la complejidad del código, aunque se mantuvo la estructura modular de $16 \times 16$ para trabajo a futuro.
\subsubsection{Texturizado}

Finalmente, el terreno fue texturizado automáticamente aplicando un material autogenerado y semi-autocoloreado según la altura (mar, pasto, nieve). En la  \cref{fig:inspector_colores} se muestra el script desde la vista del inspector dentro del objeto \textit{Terreno\_Colombiav2} en donde se evidencia la interfaz hecha para manejar las capas por color y alturas.

\begin{figure}[H]
    \centering
    \includegraphics[width=0.8\textwidth, trim=0 380 0 0, clip]{figures/inspector_colores.png}
    \caption{Captura de la clase \textit{HeighBandsPainter} desde el inspector.}
    \label{fig:inspector_colores}
\end{figure}


La \cref{fig:mapa_unity} muestra el resultado final del terreno implementado.

\begin{figure}[H]
\centering
\includegraphics[width=1\textwidth]{figures/mapa_unity.jpeg}
\caption{Implementación del entorno utilizando el sistema de Terrenos.}
\label{fig:mapa_unity}
\end{figure}

\subsection{Desarrollo del Sistema Ferroviario}

La implementación del sistema ferroviario se subdividió en tres retos técnicos principales: la generación procedural de las vías, la gestión de estructuras complejas y la sincronización del movimiento de los trenes. A continuación se detalla la solución técnica para cada componente.

\subsubsection{Generación de Vías y Optimización}

Para resolver la lógica matemática del trazado y el seguimiento de los vagones, se utilizó la librería \textit{Dreamteck Splines} como base. Dado que esta herramienta ofrecía funcionalidades preexistentes, el trabajo se centró en modificar y adaptar su código fuente para que fuera compatible con los modelos de vagones propios del proyecto. Sin embargo, por razones de optimización y rendimiento, se decidió prescindir del sistema de curvado suave de mallas, optando en su lugar por una generación de vías con ángulos agresivos y tramos rectos, visualizados mediante planos texturizados que simulan el balasto y los rieles.

\subsubsection{Sistemas de Construcción y Estructuras}

La interacción para la construcción opera mediante un sistema de dos puntos: el primer clic sobre el terreno establece el nodo de inicio del spline y el segundo define el destino, momento en el cual el algoritmo interpola puntos intermedios para ajustar la vía a la topografía del suelo. El desafío mayor surgió con la implementación de estructuras para atravesar accidentes geográficos como valles, cuerpos de agua o montañas. Para solucionar esto, se desarrolló un modo de construcción estructural que ignora la adaptación al terreno intermedia, forzando una conexión recta entre los puntos. Mediante el uso de \textit{Raycasting}, el sistema analiza el entorno de este vector: si detecta terreno por encima de la vía, instancia automáticamente un túnel; si detecta vacío por debajo, genera los pilares correspondientes para formar un puente.

\begin{figure}[H]
\centering
\begin{minipage}{0.4\textwidth}
    \centering
    \includegraphics[width=\linewidth]{figures/puente.png}
    \caption{Modelo de un trazo de un puente.}
    \label{fig:puente}
\end{minipage}\hfill
\begin{minipage}{0.4\textwidth}
    \centering
    \includegraphics[width=\linewidth]{figures/tunel.png}
    \caption{Modelo de un trazo de túnel.}
    \label{fig:tunel}
\end{minipage}
\end{figure}

Además, se implementó un sistema de previsualización al cual nombramos \textit{PreviewSpline}, con el fin de poder mostrar al jugador una previsualización del trazo o camino que va a realizar, además de ponerse de colores verde o rojo dependiendo si pueden construir en esa zona o no. Esta ayuda visual mejoró drásticamente la claridad e intuición a la hora de construir vías férreas sobre el terreno.

\subsubsection{Lógica y Datos de los Trenes}

Una vez definida la infraestructura, el sistema permite la selección de la locomotora y los vagones, cuyo costo en recursos varía según su tipología. La arquitectura de datos de estas unidades se gestionó mediante \textit{Scriptable Objects}, lo que permitió modularizar estadísticas inmutables como el precio, la salud, la velocidad máxima y el tipo de carga. Finalmente, toda esta lógica se vinculó a la interfaz de usuario (Canvas) del nivel para permitir la interacción en tiempo real, añadiendo detalles de retroalimentación visual como sistemas de partículas que se activan cuando el tren recibe daño durante el recorrido.

\begin{figure}[H]
\centering
\begin{minipage}{0.4\textwidth}
    \centering
    \includegraphics[width=\linewidth]{figures/tren_modelo.jpeg}
    \caption{Modelado de locomotoras.}
    \label{fig:tren_modelo}
\end{minipage}\hfill
\begin{minipage}{0.4\textwidth}
    \centering
    \includegraphics[width=\linewidth]{figures/trenExplota.png}
    \caption{Sistema de partículas de daño.}
    \label{fig:trenExplota}
\end{minipage}
\end{figure}



\subsection{Zonas de daño y prohibición}
Para garantizar la coherencia geográfica y aumentar la dificultad, se implementaron volúmenes de control invisibles en el mapa. La zona de prohibición impiden la construcción directa del jugador, es un plano gigante que cubre todo el terreno pero está al nivel del mar, esto con el objetivo de que no pudieran construir sobre el mar. Mientras que las zonas de daño detectan colisiones físicas del tren y le reducen la vida periódicamente.

La implementación de estas zonas de daño se diseñó bajo un principio de modularidad y separación de datos. El componente lógico principal, denominado \texttt{ZonaDeDaño}, actúa únicamente como un contenedor referencial que no almacena valores directos. En su lugar, consume un activo de datos externo de tipo \textit{Scriptable Object} llamado \texttt{TipoZonasDaño}. Este activo encapsula las propiedades inmutables de la amenaza, definiendo atributos como el nombre identificador y la magnitud del daño flotante.

\begin{figure}[H]
\centering
\includegraphics[width=0.9\textwidth]{figures/zona_daño.png}
\caption{Ejemplo de una zona de daño en el terreno.}
\label{fig:zona_daño}
\end{figure}

Esta estructura permite al equipo de diseño equilibrar la dificultad creando variaciones de peligros (ej. ``Derrumbe Ligero'' vs ``Zona Volcánica'') editando un solo archivo de conción, el cual se propaga automáticamente a todas las instancias en el nivel. Aunque cabe mencionar, que se decidió no usar más de 1 tipo de zona de daño. Esta decidisión fue en consecuencia del temor a aumentarle demasiado la dificultad al juego.

Finalmente, la materialización de estas zonas en la escena se logra mediante la composición de dos elementos: un \textit{Mesh Collider} condo como disparador (\textit{Trigger}) para la detección física, y un proyector \textit{Decal} de color rojo semitransparente que renderiza una advertencia visual sobre la topografía irregular del terreno, permitiendo al jugador identificar el peligro intuitivamente.

\subsection{Managers}
La arquitectura del juego se orquesta a través de una serie de clases controladoras (\textit{Managers}) que centralizan la lógica de negocio y mantienen el estado global de la aplicación, estas son las principales:

\subsubsection{Economía}

La gestión financiera es administrada por la clase \texttt{EconomiaManager}, la cual opera en tiempo real durante el ciclo \texttt{Update()} para recalcular el costo total de la infraestructura desplegada. Siguiendo el patrón de diseño orientado a datos, este componente no almacena los precios de los activos internamente; en su lugar, consume una referencia externa al \textit{Scriptable Object} \texttt{CondicionesGlobales}, lo que permite ajustar la inflación o el costo de los materiales (vías por metro, precio por puente/túnel) en un solo archivo para todo el juego.

El cálculo del saldo disponible se realiza mediante la siguiente sumatoria:

\begin{equation}
    S_{remantente} = P_{inicial} - (C_{vías} + C_{estructuras} + C_{tren})
\end{equation}

Un aspecto fundamental de la experiencia de usuario (UX) en esta implementación es la ausencia de bloqueos restrictivos. A diferencia de los simuladores tradicionales que impiden la construcción al llegar a cero, el sistema permite que el jugador continúe extendiendo la vía incluso si el saldo entra en valores negativos. Esta decisión de diseño prioriza la finalización del nivel y la libertad creativa; sin embargo, incurrir en deuda tiene una consecuencia punitiva en la fase de evaluación, resultando en la pérdida automática de la \textit{Estrella de Economía} (más adelante se detallará sobre esto) y reduciendo drásticamente el puntaje final.

\subsubsection{Sistema de Guardado}

La persistencia local de datos es gestionada por la clase \texttt{SaveSystem}, un componente que implementa el patrón \textit{Singleton} para garantizar una única instancia de acceso a archivos durante la ejecución. El sistema serializa el estado del juego en formato JSON y lo almacena en el directorio persistente del dispositivo (\texttt{Application.persistentDataPath}), asegurando que el progreso sobreviva al cierre de la aplicación.

La lógica de guardado no es destructiva; utiliza un algoritmo de actualización condicional detallado a continuación:

\begin{enumerate}
    \item Progresión Lineal: El sistema incluye el método \texttt{IsLevelUnlocked}, el cual verifica la existencia de datos completados del nivel $N-1$ para permitir el acceso al nivel $N$.
    \item Validación de Mejor Puntuación: Al finalizar un nivel, el sistema compara el puntaje obtenido con el almacenado previamente. El archivo solo se actualiza si el jugador ha superado su propio récord (\textit{High Score}).
    \item Fusión de Estrellas: Las condiciones de victoria (estrellas) se guardan de manera acumulativa. Si el jugador obtuvo la ``Estrella de Tiempo'' en un intento anterior y la ``Estrella de Economía'' en el actual, el sistema fusiona ambos resultados mediante una operación lógica OR, permitiendo completar el nivel en múltiples intentos.
\end{enumerate}

\subsubsection{Nivel}

La clase \texttt{NivelSpline} actúa como el orquestador central o máquina de estados finitos de la sesión de juego. Su diseño sigue un patrón orientado a datos (\textit{Data-Driven}), delegando la conción de reglas (tiempo límite, presupuesto inicial) y fórmulas matemáticas a dependencias externas inyectadas mediante \textit{Scriptable Objects} (\texttt{CondicionesNivel} y \texttt{CondicionesGlobales}). Esto permite ajustar el balance del juego sin modificar el código fuente.

El ciclo de vida gestionado por este componente se divide en tres etapas críticas:

\begin{itemize}
    \item Gestión Temporal: Distingue entre el tiempo de ``Planeación'' (desde la carga de la escena) y el tiempo de ``Simulación'' (activación del tren), lo cual es vital para no penalizar al jugador mientras diseña la vía.
    \item Validación de Ruta: Implementa un sistema de control de puntos de paso (\textit{Checkpoints}). Mediante la detección de colisiones, verifica que el tren recorra la ruta geográfica obligatoria en el orden correcto, impidiendo atajos no válidos.
    \item Resolución de Estado: Al finalizar el recorrido, centraliza la información de los demás gestores (economía, integridad del tren y tiempo), ejecuta el algoritmo de puntuación descrito anteriormente y orquesta la persistencia de datos, enviando los resultados tanto al sistema de guardado local como al servicio de telemetría en la nube.
\end{itemize}

\begin{figure}[H]
    \centering
    \includegraphics[width=0.6\textwidth]{figures/pantalla_fallar.png}
    \caption{Interfaz de retroalimentación tras un evento de derrota.}
    \label{fig:estado_nivel}
\end{figure}

\subsubsection{Telemetría}

El componente \texttt{AnalyticsManager} es el responsable de la comunicación Cliente-Servidor. A diferencia del guardado local, este sistema no almacena datos en el dispositivo, sino que los empaqueta y envía a una API REST alojada en \textit{Google Cloud Functions}.

Su funcionamiento se divide en tres procesos concurrentes:

\begin{itemize}
    \item Captura de Interacción (Heatmaps): Durante el ciclo de juego (\texttt{Update}), el script registra las coordenadas $(x, y)$ de cada clic realizado por el usuario en la interfaz o el terreno. Estos datos se almacenan temporalmente en una lista de objetos \texttt{ClickPos} y se adjuntan al reporte final, lo que permite la futura generación de mapas de calor para analizar el comportamiento del jugador.
    
    \item Registro de Intentos: Al finalizar una partida (ya sea por victoria, derrota o abandono), se instancia un objeto \texttt{LevelLogPayload}. Este paquete de datos contiene métricas precisas como la duración exacta de la sesión, el presupuesto consumido, la causa específica del fallo y la marca temporal (\textit{timestamp}).
    
    \item Transmisión Asíncrona: Para evitar congelamientos en la interfaz de usuario debido a la latencia de red, el envío de datos se realiza mediante Corrutinas de Unity (\texttt{IEnumerator}). Se utiliza la clase \texttt{UnityWebRequest} con el método POST para transmitir el JSON serializado, manejando internamente los encabezados y la confirmación de recepción del servidor.
\end{itemize}

\subsubsection{Tutorial}

El sistema de aprendizaje no se implementó como una escena aislada, sino como una capa lógica transversal gestionada por la clase \texttt{TutorialManager}. Este componente funciona como una máquina de estados secuencial que se superpone tanto en el Menú Principal como en el primer nivel de juego, adaptando su comportamiento mediante la bandera lógica \texttt{esTutorialDeMenu}.

La arquitectura del sistema se fundamenta en tres pilares técnicos:

\begin{enumerate}
    \item Estructura de Datos Narrativa: El contenido del tutorial se define mediante una lista de objetos serializables de la clase \texttt{DialogoTutorial}. Cada elemento contiene el texto explicativo, la expresión emocional del personaje guía (``Don Rail''), una referencia opcional a un objeto de la interfaz que debe resaltarse (activando indicadores visuales) y, lo más importante, la \texttt{TutorialAction} requerida para avanzar.
    
    \begin{figure}[H]
    \centering
    \includegraphics[width=0.6\textwidth]{figures/don_rail.png}
    \caption{Personaje guía \textit{Don Raíl}.}
    \label{fig:donRail}
    \end{figure}
    
    \item Sistema de Eventos y Polling: Para validar el progreso del jugador, el gestor utiliza un enfoque híbrido.
    \begin{enumerate}
        \item \textit{Detección Activa:} En el ciclo \texttt{Update}, el script monitorea entradas de hardware específicas, como el movimiento de los ejes del ratón o las teclas WASD, para validar lecciones de control de cámara.
        \item \textit{Detección Pasiva:} Para interacciones de interfaz (UI) o mecánicas complejas (como conectar vías), otros scripts notifican que una tarea ha sido completada invocando el método público \texttt{ReportarAccion(ID)}.
    \end{enumerate}
    
    \item Gestión de Tiempos (UX): Para evitar que el tutorial avance abruptamente apenas el usuario toca un control, se implementó una corrutina de espera (\texttt{RutinaEsperarYAvanzar}). Esta lógica introduce un retraso intencional (conble, por defecto 3 segundos) después de acciones continuas como mover la cámara, permitiendo que el jugador asimile la mecánica antes de presentar el siguiente cuadro de diálogo.
\end{enumerate}

Finalmente, el sistema se integra con el módulo de persistencia (\texttt{SaveSystem}) para verificar si el tutorial introductorio ya fue completado, evitando repeticiones innecesarias en sesiones futuras.

\subsection{Menú Principal}
El punto de entrada a la experiencia es una escena ligera diseñada para ofrecer una navegación intuitiva y rápida. La interfaz gráfica, ilustrada en la  \cref{fig:menu_principal}, conserva la identidad visual \textit{low-poly} del proyecto mediante una vista previa animada del entorno ferroviario.

Funcionalmente, el menú dispone de tres puntos de interacción principales:

\begin{enumerate}
    \item Jugar: Inicia la transición hacia la escena del \textit{Selector de Niveles}, cambiando el contexto hacia el mapa geográfico de Colombia para que el usuario elija su contrato.
    
    \item Opciones: Despliega una ventana emergente (\textit{pop-up}) dedicada a la configuración del sistema. En la versión actual del prototipo, este módulo se centra exclusivamente en la gestión del \textit{AudioManager}, permitiendo al usuario ajustar de manera independiente el volumen de la música de fondo y de los efectos de sonido (SFX) mediante deslizadores (\textit{sliders}).
    
    \item Salir: Ejecuta la instrucción de finalización del ciclo de vida de la aplicación, cerrando el juego y devolviendo el control al sistema operativo.
\end{enumerate}

\begin{figure}[H]
    \centering
    \includegraphics[width=0.8\textwidth]{figures/menu_ferro.png}
    \caption{Interfaz del Menú Principal.}
    \label{fig:menu_principal}
\end{figure}

\subsection{Menú de Pausa}
Durante la ejecución de cualquier nivel, el jugador dispone de la facultad de interrumpir el ciclo de simulación pulsando la tecla \textit{Escape}. Esta acción invoca al \textit{PauseManager}, el cual detiene el tiempo de juego (estableciendo \texttt{Time.timeScale = 0}) y despliega una interfaz superpuesta que permite gestionar la sesión sin perder el progreso actual.

Como se observa en la  \cref{fig:menu_pausa}, este menú ofrece cuatro opciones funcionales:

\begin{enumerate}
    \item Reanudar: Oculta la interfaz y reactiva el cronómetro de la simulación inmediatamente.
    \item Reiniciar: Recarga la escena actual desde cero, restableciendo todas las variables de estado (presupuesto y tiempo) para permitir un nuevo intento limpio.
    \item Guía: Abre la ventana del menú de guía del juego.
    \item Opciones: Habilita una ventana emergente para ajustar el volumen de la música y los efectos de sonido en tiempo real.
    \item Salir: Aborta la misión en curso y redirige al usuario a la escena del Selector de Niveles.
\end{enumerate}

\begin{figure}[H]
    \centering
    \includegraphics[width=0.7\textwidth]{figures/menuPausa.png} 
    \caption{Interfaz del Menú de Pausa superpuesta a la simulación.}
    \label{fig:menu_pausa}
\end{figure}

\subsubsection{Guía de Juego}
Integrado dentro de las opciones del menú de pausa, se encuentra el módulo de consulta denominado \textit{Guía} ( \cref{fig:guia_juego}). Este componente actúa como una enciclopedia interactiva y estática, diseñada para reforzar el aprendizaje de las mecánicas sin obligar al jugador a abandonar la experiencia.

\begin{figure}[H]
    \centering
    \includegraphics[width=0.7\textwidth]{figures/guia.png} 
    \caption{Panel de la Guía detallando controles, mecánicas y recomendaciones.}
    \label{fig:guia_juego}
\end{figure}

\subsection{Selector de Niveles}
Para contextualizar geográficamente al jugador, la selección de misiones no se realiza mediante una lista tradicional, sino a través de un mapa interactivo de Colombia ( \cref{fig:selector}). Los marcadores sobre el territorio indican la ubicación real de los contratos ferroviarios, reforzando el componente educativo de geografía nacional.

\begin{figure}[H]
    \centering
    \includegraphics[width=0.7\textwidth]{figures/selectorNiveles.png}
    \caption{Escena de selección de niveles sobre el mapa geográfico de Colombia.}
    \label{fig:selector}
\end{figure}

\subsection{Contratos}
Cada nivel se presenta bajo la metáfora de un "Contrato Ferroviario". Antes de iniciar, el sistema presenta una ficha técnica ( \cref{fig:contrato}) que detalla el origen, los destinos, el presupuesto asignado y el tiempo límite acordado del trayecto del tren.

\begin{figure}[H]
    \centering
    \includegraphics[width=0.6\textwidth]{figures/contratoPuntuación.png}
    \caption{Ficha de contrato que establece los objetivos y el presupuesto del nivel.}
    \label{fig:contrato}
\end{figure}

\subsection{Puntuación}
El sistema de evaluación calcula el desempeño del jugador mediante un algoritmo de suma ponderada que integra cuatro variables críticas: tiempo, capacidad de carga (vagones), eficiencia presupuestal e integridad del tren. La puntuación total $P_{total}$ se obtiene mediante la Ecuación \cref{eq:puntaje_total}, la cual normaliza el resultado y aplica un factor de escala de $10^6$ para presentar el resultado como un número entero de alto valor.

\begin{equation}
    P_{total} = \left[ w_t \cdot S_{tiempo} + w_v \cdot S_{vagones} + w_p s\cdot S_{presupuesto} + w_s \cdot S_{salud} \right] \times 10^6
    \label{eq:puntaje_total}
\end{equation}

Donde $w_x$ representan los pesos o coeficientes de importancia asignados a cada factor según el diseño del nivel, y $S_x$ son los sub-puntajes normalizados calculados de la siguiente manera:

\begin{itemize}
    \item Factor Tiempo ($S_{tiempo}$) Penaliza el exceso de tiempo respecto al objetivo mediante una función inversamente proporcional:
    \begin{equation}
        S_{tiempo} = \frac{1}{1 + \frac{T_{usado}}{T_{objetivo}}}
    \end{equation}
    
    \item Factor Vagones ($S_{vagones}$) Premia el transporte de mayor carga, normalizado sobre una base de 3 vagones estándar:
    \begin{equation}
        S_{vagones} = \text{clamp}_{0}^{1} \left( \frac{N_{vagones}}{3} \right)
    \end{equation}
    
    \item Factor Presupuesto ($S_{presupuesto}$) Evalúa la precisión en el gasto, penalizando la desviación porcentual respecto al presupuesto asignado (error relativo):
    \begin{equation}
        S_{presupuesto} = \max \left( 0, \ 1 - \frac{|D_{gastado} - D_{asignado}|}{D_{asignado}} \right)
    \end{equation}
    
    \item Factor Salud ($S_{salud}$) Representa el porcentaje de integridad restante del tren al llegar a la meta:
    \begin{equation}
        S_{salud} = \frac{V_{actual}}{V_{maxima}}
    \end{equation}
\end{itemize}

\subsubsection{Sistema de Estrellas}
La puntuación numérica $P_{total}$ se complementa con un sistema de calificación cualitativa visualizado mediante iconos de estrellas. Este esquema evalúa cuatro dimensiones independientes del desempeño del jugador, permitiendo identificar áreas específicas de mejora en la gestión del proyecto ferroviario.

Los criterios de obtención para cada insignia son:

\begin{itemize}
    \item Estrella por Completado (Cumplimiento): Es la distinción base que se otorga automáticamente al superar los requisitos mínimos del contrato, es decir, conectar el origen con los destinos correspondientes.
    
    \item Estrella de Presupuesto (Eficiencia Económica): Evalúa la gestión financiera. Se obtiene únicamente si el jugador logra finalizar la obra manteniendo un saldo positivo en la billetera virtual ($Presupuesto_{final} > 0$).
    
    \item Estrella de Vida (Seguridad Operativa): Premia la integridad de la maquinaria. Para conseguirla, el tren debe llegar a la meta sin haber perdido vida por atravezar zonas de daño durante el trayecto.
    
    \item Estrella de Tiempo (Eficiencia Temporal): Se otorga al cumplir el objetivo de tiempo ideal estipulado en el contrato, lo cual exige una planificación de ruta óptima y una velocidad de locomotora adecuada.
\end{itemize}

Finalmente, la  \cref{fig:victoria} muestra la pantalla de consolidación de éxito, donde se visualiza el puntaje numérico global junto con la cantidad de estrellas obtuvidas.

\begin{figure}[H]
    \centering
    \includegraphics[width=0.7\textwidth]{figures/pantalla_victoria.png}
    \caption{Pantalla de victoria con la calificación obtenida según los cuatro criterios.}
    \label{fig:victoria}
\end{figure}

\subsection{Pruebas y Análisis de Resultados}

Para validar el prototipo como herramienta de aprendizaje y software funcional, se realizó una prueba unitaria con 28 voluntarios de la comunidad universitaria. El proceso de recolección de datos siguió un enfoque mixto: los datos cuantitativos fueron extraídos directamente de la API de informes (telemetría), mientras que los cualitativos se obtuvieron mediante una encuesta de percepción aplicada tras la sesión de juego.

\subsection{Encuesta}

Se aplicó una encuesta desarrollada en Google Forms a los 28 voluntarios que participaron en las pruebas. Este instrumento permitió contrastar el comportamiento registrado por el sistema con la experiencia subjetiva del usuario, evaluando la efectividad del videojuego como herramienta de aprendizaje GBL. La encuesta, compuesta por 30 preguntas, se enfocó en la claridad de las mecánicas y la comprensión de los retos de la ingeniería ferroviaria en Colombia.

En la \cref{fig:encuesta} se observa el formato del instrumento utilizado para la recolección de datos cualitativos.

\begin{figure}[H] \centering \includegraphics[width=0.7\textwidth]{figures/encuesta.png} \caption{Formulario de evaluación de la experiencia de usuario y aprendizaje.} \label{fig:encuesta} \end{figure}

Las preguntas que integraron el instrumento de evaluación se presentan a continuación:

\begin{enumerate}
    \item ¿En qué rango de edad se encuentra?
    \item ¿Ha jugado videojuegos antes?
    \item Antes de jugar ¿Conocía algo sobre trenes o vías férreas?  
    \item ¿Qué tan clara fue su primera experiencia al empezar el juego?
    \item ¿El estilo visual (colores, íconos) le pareció atractivo?
    \item ¿Le pareció que el mapa refleja la complejidad geográfica de Colombia (montañas, valles, ríos)?
    \item ¿Le pareció que entendió los conceptos básicos gracias al tutorial?
    \item Después de terminar el tutorial, ¿se sintió seguro/a para jugar solo/a sin necesidad de volver a jugarlo para entender?
    \item ¿Qué tan fáciles fueron de entender los controles?  
    \item ¿Qué tan intuitivo le resultó el sistema de trazado de vías?
    \item ¿Tuvo problemas con la cámara o el movimiento?  
    \item Al construir sobre terrenos montañosos, ¿le pareció que el sistema de puentes y túneles se generó de forma lógica?
    \item ¿Cómo calificaría la respuesta de los vagones al seguir la locomotora en curvas pronunciadas?
    \item ¿Le pareció que el objetivo del nivel estaba claro?  
    \item ¿Qué tan fácil es identificar el presupuesto y los recursos disponibles mientras construye?
    \item ¿Le pareció justa la manera en la que el juego decide si gana o pierde?  
    \item ¿Entendió el sistema de puntuación?
    \item ¿Qué tan divertido le pareció el juego?  
    \item ¿Qué parte fue su favorita?  
    \item ¿Qué aspectos del juego le generaron frustración?
    \item ¿Tuvo alguna dificultad para leer los textos o distinguir los elementos del mapa debido al tamaño o los colores?  
    \item ¿Notó caídas de fotogramas durante el juego (imagen congelada, cortes de imagen)?
    \item ¿Encontró algún error que detuviera su juego o algo que se viera "roto"?
    \item ¿Qué cambiaría para que la experiencia sea mejor?  
    \item ¿Encontró algún fallo durante su experiencia? por favor, cuéntenos
    \item ¿Le pareció que tenía que planear antes de construir, y no solo colocar vías al azar?  
    \item ¿Cómo calificaría el impacto de la planificación en sus resultados?
    \item Después de jugar, ¿cree que construir un tren entre ciudades lejanas en Colombia es:  
    \item ¿En qué le pareció que se iba más dinero y esfuerzo?  
    \item Tras la experiencia de juego, ¿qué reflexiones tuvo adquirió respecto a la problemática del desarrollo de infraestructura férrea en el país?
\end{enumerate}

\subsection{Análisis de Telemetría (Datos de la API)}

El uso de telemetría permitió registrar el comportamiento objetivo de los usuarios, eliminando los sesgos de la autopercepción y proporcionando una visión clara sobre la jugabilidad y la curva de dificultad.

\subsubsection{Retención y Progresión: El desafío de la curva de aprendizaje}
La  \cref{fig:retencion} muestra el volumen de jugadores únicos que accedieron a cada nivel. El registro indica que 28 usuarios iniciaron el Nivel 1. En el Nivel 2 la participación fue de 9 usuarios, en el Nivel 3 de 8 usuarios, en el Nivel 4 de 5 usuarios, y finalmente, un único jugador alcanzó el Nivel 5.

\begin{figure}[H]
\centering
\includegraphics[width=0.8\textwidth]{figures/retencion.png} 
\caption{Tasa de retención de jugadores únicos por nivel.}
\label{fig:retencion}
\end{figure}

\subsubsection{Balance de Dificultad: Victorias frente a Derrotas}

La  \cref{fig:tasa_exito} detalla la proporción de intentos exitosos y fallidos. El Nivel 1 presenta una tasa de derrota del 71.4\%. En el Nivel 2, el porcentaje de éxito fue del 66.7\%, mientras que en el Nivel 3 las derrotas representaron el 69.2\%. Los niveles finales (4 y 5) registraron tasas de fracaso del 89.3\% y 87.5\% respectivamente sobre el total de intentos.

\begin{figure}[H]
\centering
\includegraphics[width=1\textwidth]{figures/tasa_exito.png} 
\caption{Proporción de victorias frente a derrotas por nivel.}
\label{fig:tasa_exito}
\end{figure}

\subsection{Resultados de la Encuesta de Percepción}

La encuesta permitió evaluar si los objetivos de concientización y usabilidad definidos en el diseño se cumplieron satisfactoriamente desde la perspectiva del usuario.

\subsubsection{Usabilidad y Experiencia de Usuario (UX)}

La percepción sobre los controles fue mayoritariamente positiva, con un 81.9\% de aprobación ( \cref{fig:facilidadcontroles}). No obstante, existe un 9\% de usuarios que reportaron dificultades.

\begin{figure}[H]
\centering
\includegraphics[width=1\textwidth]{figures/facilidadcontroles.png} 
\caption{Nivel de facilidad percibida en los controles.}
\label{fig:facilidadcontroles}
\end{figure}

La claridad sobre la primera experiencia de los jugadores fue mayormente favorable, con un 54.5\% de los encuestados reportando que fue clara ( \cref{fig:primeraexperiencia}). No obstante existe un 9\% de usuarios que reportaron que era confusa o muy confusa.

\begin{figure}[H]
\centering
\includegraphics[width=1\textwidth]{figures/primeraexperiencia.png}
\caption{Respuestas a la pregunta 4}
\label{fig:primeraexperiencia}
\end{figure}

\subsubsection{Concientización y Percepción del Aprendizaje (GBL)}
Esta sección constituye el núcleo de la validación del aprendizaje. Como indica la  \cref{fig:contruisviaslejanas}, un abrumador 81.8\% de los participantes concluyó que la construcción ferroviaria en Colombia es \textit{complicada y costosa}.

\begin{figure}[H]
\centering
\includegraphics[width=0.75\textwidth]{figures/contruisviaslejanas.png}
\caption{Percepción de la complejidad constructiva tras la experiencia de juego.}
\label{fig:contruisviaslejanas}
\end{figure}

En las s\cref{fig:dineroyesfuerzo} y \cref{fig:encuestaaprendizaje} se muestran algunas de las respuestas de la preguntas 29 y 30. Esta pregunta fue hecha con énfasis en responder el objetivo del proyecto, el cual involucra saber el impacto que tuvo en los jugadores el videojuego serio.

\begin{figure}[H]
\centering
\includegraphics[width=0.9\textwidth]{figures/dineroyesfuerzo.png}
\caption{Respuestas a la pregunta 29}
\label{fig:dineroyesfuerzo}
\end{figure}

\begin{figure}[H]
\centering
\includegraphics[width=0.9\textwidth]{figures/encuestaaprendizaje.png}
\caption{Reflexión final sobre los desafíos del desarrollo ferroviario en Colombia.}
\label{fig:encuestaaprendizaje}
\end{figure}

\section{Errores Detectados}
Durante la prueba unitaria, se reportaron los siguientes fallos que requieren corrección:
\begin{itemize}
    \item Cámara en límites del mapa: La cámara permite visualizar el \textit{vacío} en los bordes extremos. Se requiere implementar \textit{clamping} en las coordenadas.
    \item Colisiones fantasma: El tren detectó daño en el 2\% de las partidas sin colisión visual clara. Se deben refinar los \textit{colliders}.
    \item Renderización a larga distancia: Los ríos y el terreno no se visualizan adecuadamente al alejar la cámara; se recomienda aumentar la distancia de renderizado.
    \item Don Raíl en el menú: El personaje reaparece tras completar el tutorial, indicando que la variable \texttt{tutorialMenuCompletado} no cambia su estado a \texttt{True}.
    \item Cámara en nivel 5: La cámara inicia erróneamente en el punto final del recorrido en lugar del inicial, generando frustración.
\end{itemize}