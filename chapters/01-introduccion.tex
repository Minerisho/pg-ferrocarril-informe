% !TeX root = ../main.tex

\chapter{Introducción}

\section{Introducción}
El transporte ferroviario a lo largo de la historia ha desempeñado un papel fundamental para el desarrollo económico y logístico de los países industrializados. Colombia, a pesar de haber sido pionera en la introducción del ferrocarril en América Latina, ha presentado un estancamiento significativo a lo largo de las ultimas décadas, producto de la priorización de la infraestructura vial y de debilidades institucionales y técnicas en el sector. Según datos del ministerio de transporte, mas del 60\% de la red ferroviaria nacional se encuentra inactiva, y el modo férreo apenas representa el 0.1\% de la carga nacional movilizada.

En los últimos años, han resurgido iniciativas como el corredor de La Dorada-Chiriguaná, el Regiotram de Bogotá y el Tren de Cercanías del Valle del Cauca, que evidencian un renovado interés por revitalizar el sistema férreo colombiano. En este contexto, la aplicación de tecnologías interactivas, como los videojuegos educativos, emerge como una alternativa viable para fomentar la comprensión y el interés de las personas sobre los desafíos técnicos, económicos y sociales asociados al desarrollo ferroviario.

En este contexto, el Game Based Learning (GBL) surge como una iniciativa pedagógica prometedora, proponiendo el uso de entornos interactivos y dinámicos para promover la adquisición de conocimientos mediante la experiencia y la retroalimentación. Por su parte, los videojuegos serios se desarrollan con fines más allá del entretenimiento, integrando objetivos pedagógicos y simulaciones realistas. Estas herramientas permiten representar de forma didáctica problemas complejos del mundo real, como los que enfrenta la infraestructura férrea del país.

Este trabajo de grado tiene como objetivo el desarrollo de un prototipo de videojuego serio orientado a la comprensión de los desafíos asociados al desarrollo ferroviario en Colombia. Implementado mediante el motor grafico Unity 6 y el software de modelado Blender se busca crear una experiencia educativa que incorpore los criterios esenciales del GBL: inmersión, interacción, control del aprendiz, apoyo del aprendizaje, narrativa y evaluación.


\section{Planteamiento y justificación del problema}
Colombia, a pesar de haber sido pionera en la introducción del ferrocarril en Latinoamérica, se ha visto estancada en su desarrollo ferroviario. Tras una búsqueda en la plataforma web de la Agencia Nacional de Infraestructuras sobre los proyectos activos \autocite{aniProyectos}, de los 133 proyectos activos, sólo 4 son proyectos ferroviarios, de los cuales 2 llevan desde 1998 y 1999 en operación. Además, según datos del Ministerio de transporte de Colombia \autocite{mintransporteSurcos2024}, el 63,2\% de la red ferroviaria nacional se encuentra inactiva y, excluyendo el transporte de carbón y petróleo, el modo ferroviario apenas representa el 0,1\% de la carga nacional movilizada, lo cual evidencia una falta de interés e iniciativa por parte de los gobiernos que, a su vez, ha derivado en una priorización de la inversión pública en la infraestructura de carreteras \autocite{mintransporteDatosCarga}.

Según la monografía \textit{Desafíos del transporte ferroviario de carga en Colombia}, la dificultad en desarrollar efectivamente un transporte ferroviario en Colombia es debido a factores como los conflictos políticos, el manejo de los recursos y de la debilidad ingenieril del sector. Específicamente se menciona: “La institucionalidad del sector presenta hoy debilidades en materia de planeación y política ferroviaria; ingeniería ferroviaria conceptual…” \autocite[p.~17]{iabdDesafios}.

Actualmente, hay que considerar que recientemente la nación está intentando reiniciar el desarrollo de infraestructuras ferroviarias. Algunos de los proyectos que evidencian este reciente interés son: la firma del contrato de concesión, la primera Asociación Público-Privada (APP) del país, correspondiente al corredor de La Dorada–Chiriguaná \autocite{mintransporteAPP2025}; el inicio de las obras del Regiotram de Occidente, que beneficiará a Bogotá y a sus municipios aledaños \autocite{bogotaRegiotram2025}; y el megaproyecto del Tren de Cercanías del Valle del Cauca \autocite{valoraTrenValle2024}, que busca conectar Cali con Jamundí, Yumbo y Palmira.

Aun así, ante la carencia de herramientas que refuercen la planeación y la ingeniería ferroviaria, resulta oportuno desarrollar recursos que impulsen el aprendizaje y la concientización sobre los desafíos de este sector.

Los videojuegos en su mayoría se enfocan en el entretenimiento del jugador, sin embargo, existe la posibilidad de diseñar juegos educativos, tal es el caso de un estudio descrito en el artículo “Examining the characteristics of game-based learning: A content analysis and design framework” \autocite{gblFrameworkExamining} en donde, a partir del análisis y la codificación de 194 artículos diferentes sobre juegos educativos, se definió un marco de diseño de un \textit{Game-Based Learning} (GBL), o aprendizaje basado en un juego.

Un GBL se compone por dos grupos de características, las primarias y secundarias, siendo las primarias los aspectos que se consideran esenciales para un juego educativo \autocite{gblFrameworkExamining}: Criterios de inmersión, interacción, control del aprendiz, apoyo al aprendizaje, narrativa y evaluación.

Además, según el libro \textit{Serious Games: Mechanisms and Effects}, se define un videojuego serio como “cualquier forma de software de juego interactivo basado en computadora (…) y que se ha desarrollado con la intención de ser más que entretenimiento.” \autocite[p.~6]{seriousGamesMechanisms}. Aspecto que aplica para el caso de estudio de este proyecto, ya que se tiene planteado, además de entretener, educar sobre los desafíos del desarrollo ferroviario en Colombia.

Al respecto, existen diversos antecedentes de videojuegos de simulación enfocados en diseñar y construir sistemas férreos, como son el \textit{Railway Empire} \autocite{railwayEmpire} ambientado en Estados Unidos entre siglo XIX y XX, \textit{Railroad Tycoon 3} \autocite{railroadTycoon3} basado en escenarios para recrear sistemas férreos importantes en la historia de la humanidad y \textit{Train Valley 2} \autocite{trainValley2} enfocado en la resolución de puzles; sin embargo, ninguno de los videojuegos mencionados tiene propósitos más allá del entretenimiento, objetivo que sí se busca en este proyecto de grado.

Tras una búsqueda con el motor Google Academics y usando la ecuación: (“Colombia” AND “Sistema Férreo” AND (“Videojuego Serio” OR “Videojuego de simulación” OR “Videojuego educativo”)), se encontró solamente una tesis de grado que tenía como objetivo “Investigar sobre interfaces hápticas para aplicaciones 3D, así como analizar su interactividad y usabilidad en una aplicación interactiva para la maqueta de transportes ESPOCH.” \autocite{tesisInterfacesHapticas}, en el que no es un aspecto suficiente para que sea considerado el desarrollo de un videojuego serio, ya que en ningún momento describe el proceso de desarrollo de un videojuego, solo se menciona como un ejemplo. En ese orden de ideas, se puede concluir que no se encontró ningún videojuego serio de sistemas ferroviarios ambientado en Colombia.

Por lo anterior, se formula la siguiente pregunta de investigación:

\textbf{¿Cuál debería ser la estructura de un prototipo de videojuego serio sobre los desafíos del desarrollo del sistema ferroviario en Colombia y cómo se implementaría para que tenga en cuenta los criterios de inmersión, interacción, control del aprendiz, apoyo al aprendizaje, narrativa y evaluación del marco GBL?}

En el orden de ideas de lo manifestado, se puede considerar viable una iniciativa en el desarrollo de un prototipo de software para la educación del sector ferroviario, por lo tanto, se propone implementar un primer prototipo de un videojuego serio de sistemas ferroviarios en Colombia, con el fin de concientizar la comprensión de algunos de los desafíos que conllevaría la planeación y el desarrollo de la infraestructura férrea en el país.

Se espera que el prototipo del videojuego serio sobre los desafíos del sistema ferroviario en Colombia incorpore las características primarias del marco GBL como la evaluación (cómo se mide el aprendizaje), la inmersión (cuán envolvente es el entorno del juego), la interacción (cómo se comunica el jugador con el juego o entre jugadores), el control del aprendiz (el grado de autonomía que tiene el usuario), el apoyo al aprendizaje (tutoriales y ayudas) y la narrativa (la historia o contexto que da sentido a la experiencia). Estas características se usarán para “ayudar a trazar rutas de aprendizaje más personalizadas, motivadoras y potencialmente efectivas para el aprendizaje.” \autocite[p.~10]{gblHandbook2019}.

El prototipo tendrá una prueba piloto dirigida a estudiantes UIS, con el objetivo de evaluar la experiencia del jugador y el desempeño del software. La prueba se orientará a determinar si el prototipo resulta agradable en su uso y se encuentre libre de errores, permitiendo identificar áreas de mejora para quienes deseen dar continuidad a este proyecto.

\section{Objetivos}

\subsection{Objetivo General}
Desarrollar un prototipo de un videojuego serio para la comprensión de algunos de los desafíos que conllevaría el mejoramiento de la infraestructura férrea en Colombia, usando GBL, Unity 6, Blender y RUP adaptada con elementos de Kanban.

\subsection{Objetivos Específicos}
\begin{enumerate}[label=\arabic*.]
  \item Definir las historias de usuario del videojuego serio a partir del marco GBL para el apoyo de la comprensión de algunos de los desafíos del desarrollo férreo.
  \item Diseñar los componentes que integran el videojuego serio tales como los sistemas de evaluación, inmersión, interacción y control del jugador a partir del contexto de sistemas ferroviarios en Colombia.
  \item Implementar el videojuego serio integrando un sistema interactivo de un mapa 3D, así como eventos ambientales, económicos y sociales usando Unity 6 y Blender.
  \item Evaluar el videojuego serio realizando pruebas de funcionalidad y usabilidad.
\end{enumerate}
