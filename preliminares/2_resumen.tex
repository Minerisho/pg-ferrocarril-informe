% 2_resumen.tex

% --- PARTE EN ESPAÑOL ---
\begin{center}
    \textbf{\Large Resumen} % Título manual porque estamos dentro del mismo archivo
\end{center}
\vspace{1em}
\textbf{Titulo: }  PROTOTIPO DE UN VIDEOJUEGO SERIO SOBRE
SISTEMAS FERROVIARIOS EN COLOMBIA.

\noindent\textbf{Autor: } Miguel Ángel Plata Rodriguez, Mateo Salazar Serrano.

\noindent\textbf{Palabras Clave:} Videojuego Serio, Unity, GBL, Ferrocarriles, Colombia, Simulación.

\vspace{1em}
\noindent\textbf{Descripción: } La infraestructura ferroviaria en Colombia presenta un estancamiento crítico con más del 60 \% de su red inactiva, lo que motiva la creación de herramientas pedagógicas que fomenten la valoración social del sistema. Mediante el motor gráfico Unity 6, se desarrolló un prototipo de videojuego serio fundamentado en el marco de Game-Based Learning (GBL) para concientizar sobre los desafíos técnicos y económicos del sector. La implementación integró los seis criterios esenciales del GBL: la inmersión mediante un entorno geográfico basado en datos reales; la interacción y el control del aprendiz a través de libertad de control del jugador para dibujar vías en un terreno abierto; el apoyo al aprendizaje mediante mentoría a través de calificaciones; una narrativa orientada al cumplimiento de contratos estatales con un personaje llamado Don Raíl; y una evaluación técnica gestionada por una API de telemetría. Los resultados de la prueba piloto evidenciaron una concientización de la problemática ferroviaria Colombiana, logrando que el 81,8 \% de los participantes identificara la dificultad en la infrastructura férrea como una tarea compleja y costosa debido al impacto de la topografía y el uso necesario de túneles y puentes. Adicionalmente, los datos de telemetría validaron la representación de la dificultad real de la ingeniería ferroviaria nacional. Así, el videojuego logra trascender el entretenimiento al transformar la percepción del usuario hacia una visión técnica y estratégica, cumpliendo con el objetivo de favorecer el aprendizaje sobre los retos estructurales del transporte férreo en el país.
% --- PARTE EN INGLÉS (ABSTRACT) ---
\clearpage % Salto de página forzado para que el Abstract quede en hoja nueva
\begin{center}
    \textbf{\Large Abstract} 
\end{center}
\vspace{1em}
\textbf{Title: } PROTOTYPE OF A SERIOUS VIDEO GAME ABOUT RAILWAY SYSTEMS IN COLOMBIA.

\noindent\textbf{Author: } Miguel Ángel Plata Rodríguez, Mateo Salazar Serrano.

\noindent\textbf{Keywords:} Serious Game, Unity, GBL, Railways, Colombia, Simulation.

\vspace{1em}
\noindent\textbf{Description: } Colombian railway infrastructure presents a critical stagnation with over 60\% of its network inactive, which motivates the creation of pedagogical tools that foster the social valuation of the system. Using the Unity 6 game engine, a serious game prototype was developed grounded in the Game-Based Learning (GBL) framework to raise awareness about the technical and economic challenges of the sector. The implementation integrated the six essential criteria of GBL: immersion through a geographic environment based on real data; interaction and learner control through the player's freedom to draw tracks on open terrain; learning support through mentorship via ratings; a narrative oriented towards the fulfillment of state contracts with a character named Don Raíl; and a technical assessment managed by a telemetry API. The results of the pilot test evidenced an awareness of the Colombian railway issues, with 81.8\% of participants identifying the intricacy of rail infrastructure as a complex and costly task due to the impact of topography and the necessary use of tunnels and bridges. Additionally, telemetry data validated the representation of the real difficulty of national railway engineering. Thus, the video game succeeds in transcending entertainment by transforming the user's perception towards a technical and strategic vision, fulfilling the objective of promoting learning about the structural challenges of rail transport in the country.