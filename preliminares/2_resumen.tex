% 2_resumen.tex

% --- PARTE EN ESPAÑOL ---
\begin{center}
    \textbf{\Large Resumen} % Título manual porque estamos dentro del mismo archivo
\end{center}
\vspace{1em}
\textbf{Titulo: }  PROTOTIPO DE UN VIDEOJUEGO SERIO SOBRE
SISTEMAS FERROVIARIOS EN COLOMBIA

\noindent\textbf{Autor: } Miguel Ángel Plata Rodriguez, Mateo Salazar serrano

\noindent\textbf{Palabras Clave:} Videojuego Serio, Unity, GBL, Ferrocarriles, Colombia, Simulación.

\vspace{1em}
\noindent\textbf{Descripción: } La infraestructura ferroviaria en Colombia enfrenta un estancamiento crítico, con más del 60 por ciento de su red inactiva, lo que limita el desarrollo logístico del país y genera una baja valoración social del sistema. Para abordar esta problemática, se desarrolló un videojuego serio en el motor Unity 6, diseñado específicamente para concientizar sobre los desafíos técnicos y económicos del sector mediante el enfoque de Game-Based Learning (GBL), que integra los seis criterios esenciales del GBL para garantizar una experiencia educativa efectiva: la inmersión se logra mediante un entorno tridimensional escalado con datos geográficos reales de Colombia; la interacción y el control del aprendiz permiten al usuario diseñar rutas ferroviarias con curvas de Bézier y gestionar presupuestos limitados; el apoyo al aprendizaje se facilita a través del personaje guía "Don Raíl"; la narrativa sumerge al jugador en el cumplimiento de contratos estatales; y la evaluación se gestiona mediante una API de telemetría que almacena datos sobre el desempeño de los jugadores para que posteriormente sean analizados por un tercero. Al exponer al jugador a dilemas reales de la ingeniería y la planeación nacional, el videojuego trasciende el entretenimiento, fomentando una comprensión profunda de las dificultades que impiden la reactivación férrea. Pruebas individuales con personas de la comunidad universitaria validaron la capacidad de la herramienta para generar una mayor conciencia sobre la importancia estratégica de este sistema de transporte en el contexto nacional.
% --- PARTE EN INGLÉS (ABSTRACT) ---
\clearpage % Salto de página forzado para que el Abstract quede en hoja nueva
\begin{center}
    \textbf{\Large Abstract} % Título manual porque estamos dentro del mismo archivo
\end{center}

\vspace{1em}

Resumen in inglish

\vspace{1em}
\noindent\textbf{Keywords:} Serious Game, Unity, GBL, Railways, Colombia, Simulation.